\documentclass{simple}

\title[Open source în mediul academic]{Open source în mediul academic}
\subtitle{Studiu de caz -- Universitatea Politehnica din București}
\institute{Întâlnirile lunare RLUG -- August 2010}
\author[Răzvan Deaconescu]{Răzvan Deaconescu\\
	razvan.deaconescu@cs.pub.ro}
\date{12 august 2010}

\begin{document}

\frame{\titlepage}

% NB: Secțiunile nu sunt marcate vizual, ci doar apar în cuprins
\section{Context}

% Titlul unui frame se specifică fie în acolade, imediat după \begin{frame},
% fie folosind \frametitle
\begin{frame}{Context}
	\begin{itemize}		% Just like normal LaTeX
		\item asistent universitar
		\item Catedra de Calculatoare
			\begin{itemize}
				\item Facultatea de Automatică și Calculatoare
					\begin{itemize}
						\item Universitatea Politehnica din București
					\end{itemize}
			\end{itemize}
	\end{itemize}
\end{frame}

\frame{\tableofcontents}

\section{Administrare}

\frame{\tableofcontents[currentsection]}

\begin{frame}{Servere}
	\begin{itemize}
		\item csr.cs.pub.ro (router), Debian (2, 0)
		\item \textbf{mail.cs.pub.ro} (mail), Ubuntu (2, 300)
		\item \textbf{mamba.cs.pub.ro} (HN în OpenVZ), Debian (2, 0)
		\item \textbf{koala.cs.pub.ro} (proiecte, container), Debian (4, 35)
		\item \textbf{swarm.cs.pub.ro} (utilizatori, container), Debian
		(9, 200)
		\item \textbf{elf.cs.pub.ro} (didactic, container), Debian (4, 30)
		\item \textbf{cursuri.cs.pub.ro} (didactic), Debian (3, 50)
	\end{itemize}
\end{frame}

\begin{frame}{Servicii}
	\begin{itemize}
		\item Web: Apache
		\item Mail: Postfix
		\item SSH
		\item LDAP: OpenLDAP (ldap.grid.pub.ro, swarm.cs.pub.ro)
		\item Mailman
		\item MySQL
	\end{itemize}
\end{frame}

\begin{frame}{Servicii pentru utilizatori}
	\begin{itemize}
		\item wikis (DokuWiki, MediaWiki, PmWiki)
		\item Wordpress
		\item Planeta CS -- \url{http://planet.cs.pub.ro/}
	\end{itemize}
\end{frame}

\section{Didactic și studenți}

\frame{\tableofcontents[currentsection]}

\begin{frame}{Legendă}
	\begin{itemize}
		\item privire pe ani de studiu
		\item \textbf{Utilizarea sistemelor de operare}
			\begin{itemize}
				\item heavy open source connection
			\end{itemize}
		\item \textit{Paradigme de programare}
			\begin{itemize}
				\item some open source connection
			\end{itemize}
	\end{itemize}
\end{frame}

\begin{frame}{But first \ldots Moodle}
	\begin{itemize}
		\item \url{http://moodle.org/} -- open-source community-based tools
		for learning
		\item implementat la nivelul UPB
		\item Catedra de Calculatoare -- \url{http://curs.cs.pub.ro/}
		\item portal + utilitare de susținere a cursurilor (upload, forumuri,
		assignments, feedback, calendar, online quizes)
	\end{itemize}
\end{frame}

%\begin{frame}{Anul I}
%	\begin{itemize}
%		\item \textbf{Utilizarea sistemelor de operare}
%		\item \textit{Programarea calculatoarelor}
%		\item \textit{Structuri de date}
%		\item \textit{Metode numerice}
%		\item \textit{Electrotehnică}
%	\end{itemize}
%\end{frame}

\begin{frame}{Anul I}
	\begin{itemize}
		\item \textbf{Utilizarea sistemelor de operare}
			\begin{itemize}
				\item introducere în SO
				\item full-fledged Linux lab (Ubuntu)
				\item shell, sisteme de fișiere, comenzi de bază, scripting,
				editoare (Vim) etc.
				\item \url{http://elf.cs.pub.ro/uso/}
				\item ``Introducere în sisteme de operare''
				\url{http://books.google.com/books?id=_JFGzyRxQGcC}
			\end{itemize}
		\item \textit{Programarea calculatoarelor, Structuri de date}
			\begin{itemize}
				\item noțiuni de progrmare de bază -- C
				\item Linux, GCC, Makefile, Codeblocks
			\end{itemize}
	\end{itemize}
\end{frame}

\begin{frame}{Anul I (2)}
	\begin{itemize}
		\item \textit{Metode numerice}
			\begin{itemize}
				\item calcul numeric (math stuff)
				\item GNU Octave
			\end{itemize}
		\item \textit{Electrotehnică}
			\begin{itemize}
				\item circuite electrice
				\item SPICE
			\end{itemize}
	\end{itemize}
\end{frame}

%\begin{frame}{Anul II}
%	\begin{itemize}
%		\item \textbf{Protocoale de comunicație}
%		\item \textit{Proiectarea algoritmilor}
%		\item \textit{Paradigme de programare}
%	\end{itemize}
%\end{frame}

\begin{frame}{Anul II}
	\begin{itemize}
		\item \textbf{Protocoale de comunicație}
			\begin{itemize}
				\item laborator de network/socket programming
				\item Linux, GCC, Makefile
				\item Wireshark, ifconfig, host
			\end{itemize}
		\item \textit{Proiectarea algoritmilor}
			\begin{itemize}
				\item algoritmi :-)
				\item Linux, GCC, Makefile
			\end{itemize}
		\item \textit{Paradigme de programare}
			\begin{itemize}
				\item Scheme, Haskell, CLIPS, Prolog
				\item DrScheme, GHC, SWI-Prolog
				\item \url{http://elf.cs.pub.ro/pp/}
			\end{itemize}
	\end{itemize}
\end{frame}

%\begin{frame}{Anul III}
%	\begin{itemize}
%		\item \textbf{Rețele locale}
%		\item \textbf{Algoritmi de prelucrare distribuită}
%		\item \textbf{Sisteme de operare}
%		\item \textit{Arhitectura sistemelor de calcul}
%		\item \textit{Proiectarea cu microprocesoare}
%	\end{itemize}
%\end{frame}

\begin{frame}{Anul III}
	\begin{itemize}
		\item \textbf{Rețele locale}
			\begin{itemize}
				\item network design and administration
				\item Linux (Debian)
				\item SSH, iptables, iproute, Apache, BIND, Postfix, Courier
				IMAP, tcpdump
				\item \url{http://elf.cs.pub.ro/rl/}
				\item Rețele locale --
				\url{http://books.google.com/books?id=GdF_3ttxnRIC}
			\end{itemize}
		\item \textbf{Algoritmi paraleli și distribuiți}
			\begin{itemize}
				\item middleware tools, MPI, OpenMP, multithreaded programming
				\item MPICH, GCC, Java
			\end{itemize}
	\end{itemize}
\end{frame}

\begin{frame}{Anul III (2)}
	\begin{itemize}
		\item \textbf{Sisteme de operare}
			\begin{itemize}
				\item system programming (Linux, Windows)
				\item procese, fișiere, thread-uri, IPC, I/O, semnale,
				gestiunea memoriei
				\item Linux, Makefile, GNU Toolchain, GDB, valgrind, Cygwin
				\item \url{http://elf.cs.pub.ro/so}
			\end{itemize}
		\item \textit{Arhitectura sistemelor de calcul}
			\begin{itemize}
				\item high performance computing
				\item Python, GCC, Eclipse
			\end{itemize}
		\item \textit{Proiectarea cu microprocesoare}
			\begin{itemize}
				\item lipeli, lucru manual, programare de microcontrollere
				\item AVR GCC, AVRDUDE
				\item \url{http://elf.cs.pub.ro/pm/wiki/}
			\end{itemize}
	\end{itemize}
\end{frame}

%\begin{frame}{Anul IV}
%	\begin{itemize}
%		\item \textit{Arhitecturi paralele de calcul}
%		\item \textit{Sisteme incorporate}
%		\item \textbf{Sisteme de programe pentru rețele de calculatoare}
%		\item \textbf{Compilatoare}
%		\item \textbf{Sisteme de operare 2}
%		\item \textit{Inteligență artificială}
%		\item \textit{Interfețe om calculator}
%		\item \textit{Managementul proiectelor software}
%		\item \textit{Programare web}
%		\item \textit{Instrumente de dezvoltare a programelor}
%	\end{itemize}
%\end{frame}

\begin{frame}{Anul IV (2)}
	\begin{itemize}
		\item \textit{Arhitecturi paralele de calcul}
			\begin{itemize}
				\item programare paralelă: MPI, OpenMP, multithreaded
				programming
				\item Open MPI, GCC, Posix Threads
			\end{itemize}
		\item \textit{Sisteme incorporate}
			\begin{itemize}
				\item embedded programming
				\item ATNGW100 network gateway (embedded Linux)
				\item \url{http://elf.cs.pub.ro/si/doku.php}
			\end{itemize}
		\item \textbf{Sisteme de programe pentru rețele de calculatoare}
			\begin{itemize}
				\item middleware: CORBA, RPC, MPI
				\item Java, Linux, GCC, MPICH
			\end{itemize}
	\end{itemize}
\end{frame}

\begin{frame}{Anul IV (3)}
	\begin{itemize}
		\item \textbf{Compilatoare}
			\begin{itemize}
				\item back-end, front-end, alocarea registrelor
				\item SPIM (emulator de MIPS), ANTLR, Java
				\item \url{http://cs.pub.ro/~cpl/}
			\end{itemize}
		\item \textbf{Sisteme de operare 2}
			\begin{itemize}
				\item kernel programming / driver development (Linux, Windows)
				\item \url{http://elf.cs.pub.ro/so2/}
			\end{itemize}
	\end{itemize}
\end{frame}

\begin{frame}{Anul IV (4)}
	\begin{itemize}
		\item \textit{Inteligență artificială}
			\begin{itemize}
				\item SWI-Prolog, DrScheme
				\item \url{http://turing.cs.pub.ro/ia_09/}
			\end{itemize}
		\item \textit{Interfața om calculator}
			\begin{itemize}
				\item crearea și popularizarea unui site (în general, se
				pornește de la un CMS open-source)
			\end{itemize}
		\item \textit{Managementul proiectelor software}
			\begin{itemize}
				\item Redmine, Dia, Subversion, Git
				\item \url{http://elf.cs.pub.ro/~mps/wiki/}
			\end{itemize}
		\item \textit{Programare web}
			\begin{itemize}
				\item PHP, JSP, JS, AJAX, Tomcat
				\item \url{http://elf.cs.pub.ro/pw/}
			\end{itemize}
		\item \textit{Instrumente de dezvoltare a programelor}
			\begin{itemize}
				\item Subversion, Eclipse, Java, Ant, Tomcat, Apache Axis
				\item \url{http://elf.cs.pub.ro/idp/}
			\end{itemize}
	\end{itemize}
\end{frame}

%\begin{frame}{Master}
%	\begin{itemize}
%		\item \textbf{Planificarea și implementarea serviciilor de rețea}
%		\item \textbf{Soluții de rețea pentru ISP-uri}
%		\item \textit{Sisteme de operare avansate}
%		\item \textit{Type Systems and Functional Programming}
%		\item \textit{Distributed Algorithms}
%	\end{itemize}
%\end{frame}

\begin{frame}{Master}
	\begin{itemize}
		\item \textbf{Planificarea și implementarea serviciilor de rețea}
			\begin{itemize}
				\item bazat pe LPIC
				\item Linux
				\item administarea sistemului și serviciilor: SSH, Web,
				e-mail, database, firewalling, DNS, LDAP
				\item \url{http://elf.cs.pub.ro/~pisr/wiki/}
			\end{itemize}
		\item \textbf{Soluții de rețea pentru ISP-uri}
			\begin{itemize}
				\item network system administration
				\item Linux
				\item virtualizare (OpenVZ, VMware), network monitoring,
				Quagga, VPN, LVS, AppArmor, traffic control
				\item \url{http://elf.cs.pub.ro/srisp/wiki/}
			\end{itemize}
	\end{itemize}
\end{frame}

\begin{frame}{Master (2)}
	\begin{itemize}
		\item \textit{Sisteme de operare avansate}
			\begin{itemize}
				\item articole științifice OS/Linux-related
				\item Redmine, Subversion/Git, \LaTeX
				\item \url{http://elf.cs.pub.ro/~soa/wiki/}
			\end{itemize}
		\item \textit{Type Systems and Functional Programming}
			\begin{itemize}
				\item GHC (Haskell)
			\end{itemize}
		\item \textit{Distributed Algorithms}
			\begin{itemize}
				\item Linux, MPICH, various tools
			\end{itemize}
	\end{itemize}
\end{frame}

\section{Cercetare și proiecte}

\frame{\tableofcontents[currentsection]}

\begin{frame}{Tools of trade}
	\begin{itemize}
		\item Git, Subversion
		\item Redmine, Trac
			\begin{itemize}
				\item \url{http://koala.cs.pub.ro/redmine/}
				\item \url{http://ixlabs.cs.pub.ro/redmine/}
				\item \url{http://systems.cs.pub.ro/projects/}
				\item \url{http://ceata.org/}
				\item \url{http://rosedu.org/projects/}
			\end{itemize}
		\item DokuWiki, MediaWiki
		\item \LaTeX
	\end{itemize}
\end{frame}

\begin{frame}{NCIT cluster}
	\begin{itemize}
		\item National Center for Information Technology
		\item din 2006
		\item \url{http://cluster.grid.pub.ro/}
		\item circa 200 de sisteme
		\item Scientific Linux, Red Hat Linux
	\end{itemize}
\end{frame}

\begin{frame}{P2P-Next}
	\begin{itemize}
		\item \url{http://www.p2p-next.org/}
		\item din 2008
		\item ``next generation Peer-to-Peer (P2P) content delivery platform''
		\item ``will develop a platform that takes open source development,
		open standards, and future proof iterative integration as key design
		principles''
		\item Tribler -- \url{http://www.tribler.org/trac}
	\end{itemize}
\end{frame}

\begin{frame}{Android}
	\begin{itemize}
		\item \url{http://cluster.grid.pub.ro/index.php/android}
		\item din 2010
		\item proiect de familiarizare cu platforma Android
		\item simulatoare + Nexus One
	\end{itemize}
\end{frame}

\begin{frame}{IxLabs}
	\begin{itemize}
		\item colaborare cu Ixia
		\item din 2006
		\item proiecte de licență (open-source) + internship-uri
		\item \url{http://ixlabs.cs.pub.ro/redmine/}
	\end{itemize}
\end{frame}

\section{Altele}

\frame{\tableofcontents[currentsection]}

\begin{frame}{LPIC-1}
	\begin{itemize}
		\item curs de LPIC-1 (Linux Professional Certification)
			\begin{itemize}
				\item Junior Level Linux Professional
			\end{itemize}
		\item din octombrie 2008
		\item urmărește programa LPIC-1 --
		\url{http://www.lpi.org/eng/certification/the_lpic_program/lpic_1}
		\item prin intermediul Academiei Cisco -- \url{http://ccna.ro/}
	\end{itemize}
\end{frame}

\begin{frame}{vmchecker}
	\begin{itemize}
		\item verificare automată a temelor de casă
		\item Python, Google Web Toolkit (GWT), Linux server(, VMWare Server)
		\item Programarea algoritmilor, Sisteme de operare, Compilatoare,
		Sisteme de operare 2
		\item \url{http://github.com/vmchecker/vmchecker}
	\end{itemize}
\end{frame}

\begin{frame}{Organizații}
	\begin{itemize}
		\item Ceata -- \url{http://ceata.org/}
		\item ROSEdu -- \url{http://rosedu.org/}
	\end{itemize}
\end{frame}

\section{Întrebări}

\frame{\tableofcontents[currentsection]}

\end{document}
