\documentclass{beamer}

% Romanian Language support
\usepackage{ucs}
\usepackage[utf8x]{inputenc}
\PrerenderUnicode{aâîțșĂÎÂȚȘ}
\usepackage[english,romanian]{babel}

\usepackage{hyperref}   % use \url{http://$URL} or \href{http://$URL}{Name}
\usepackage{verbatim}
\usepackage{underscore} % underscores need not be escaped
\usepackage{booktabs}   % nice looking tables
\usepackage{array}      % column size options in tables
\usepackage[normalem]{ulem}       % for striketrough text

\mode<presentation>
%{ \usetheme{Berlin} }

% Disable useless navigation symbols.
\setbeamertemplate{navigation symbols}{}
\setbeamertemplate{footline}[frame number]

\title[Anti-eroi]{De ce ne plac anti-eroii}
\institute{ROSEdu}
\author[Răzvan Deaconescu]{Răzvan Deaconescu \\
razvan@rosedu.org}
\date{19 iunie 2018}

\begin{document}

\frame{\titlepage}

\begin{frame}{But first \ldots}
  \centering
  \Large
  \pause Doamnelor și domnilor: Snake Plissken! \\
  \normalsize
  \url{https://www.youtube.com/watch?v=r9W_Iry5wrk}
\end{frame}

\begin{frame}{Ce este un anti-erou?}
  \pause hero, anti-hero, anti-villain, villain
  \begin{itemize}
    \pause \item calități
    \pause \item defecte
    \pause \item pe vrute/nevrute, face bine
  \end{itemize}
\end{frame}

\begin{frame}{Blurred Lines}
  \centering
  \pause \textit{He who fights with monsters should be careful lest he thereby become a monster. And if thou gaze long into an abyss, the abyss will also gaze into thee.} \\
  \vspace{3mm}
  \hfill \textit{Friedrich Nietzsche} \\
  \vspace{1cm}
  \pause \textit{You either die a hero, or live long enough to see yourself become the villain.} \\
  \vspace{3mm}
  \hfill \textit{Harvey Dent (The Dark Knight)}
\end{frame}

\begin{frame}{Eroi și anti-eroi}
  \begin{itemize}
    \pause \item Superman, Batman
    \pause \item Spiderman, Venom
    \pause \item Wolverine, Deadpool, Cable
    \pause \item Captain America, Iron Man
    \pause \item Thor, Loki
    \pause \item Luke Skywalker, Han Solo
  \end{itemize}
\end{frame}

\begin{frame}{De ce ne plac anti-eroii?}
  \centering
  \Large
  \pause sunt fun \\
  \pause sunt reali, autentici \\
  \pause sunt imperfecți \\
  \pause sunt umani \\
  \pause \textbf{empatizăm cu ei}
\end{frame}

\begin{frame}{Eroi vs. anti-eroi}
  \centering
  \Large
  \pause dark \\
  \pause negativ \\
  \pause slăbiciuni \\
  \pause defecte \\
  \pause imperfecțiuni \\
  \pause \textbf{anti-modele}
\end{frame}

\begin{frame}{Reality check}
  \centering
  \Large
  \pause cu toții suntem imperfecți \\
  \pause avem părți bune și părți slabe/rele \\
  \pause calități și defecte \\
  \pause \textbf{vedem în anti-eroi imaginea noastră}
\end{frame}

\begin{frame}{Echilibru}
  \centering
  \Large
  \pause bine și rău \\
  \pause plăcere și durere \\
  \pause dulce și amar \\
  \pause yin și yang \\
  \pause \textbf{nu sunt separate, nu e bine să fie separate}
\end{frame}

\begin{frame}{Defecte, păcate, cusururi}
  \begin{itemize}
    \item furie
    \item ură
    \item violență
    \item nepăsare
    \item cinism
    \item egoism
    \item invidie
    \item mândrie
    \item lăcomie
    \item oportunism
    \item ipocrizie
    \item minciună
    \item frică
  \end{itemize}
\end{frame}

\begin{frame}{De ce defecte?}
  \centering
  \Large
  \pause let's talk about Diablo 3 \\
  \pause energie (negativ) + canalizare (pozitiv) \\
  \pause ura energizează, dragostea canalizează \\
  \pause frica energizează, curajul canalizează \\
  \pause oportunismul energizează, principialismul canalizează
\end{frame}

\begin{frame}{Evil is good}
  \centering
  \begin{figure}
    \includegraphics[width=0.7\textwidth]{img/good-to-be-bad.png}
  \end{figure}
  \tiny \url{https://fufunha.deviantart.com/art/It-s-good-to-be-bad-589653749}
\end{frame}

\begin{frame}{Apex Predators}
  \centering
  \Large
  \pause 80\% prey \\
  \pause 20\% predators \\
  \pause 1\% apex predators (kill predators)
\end{frame}

\begin{frame}{Defectele ca armură}
  \centering
  \pause \textit{Let me give you some advice bastard. Never forget what you are. The rest of the world will not. Wear it like armor, and it can never be used to hurt you.}\\
  \vspace{3mm}
  \hfill \textit{Tyrion Lannister (to Jon Snow)}
\end{frame}

\begin{frame}{Numai că \ldots}
  \centering
  \large
  \pause \textit{Remember, a Jedi's strength flows from the Force. But beware. Anger, fear, aggression. The dark side are they. Once you start down the dark path, forever will it dominate your destiny.} \\
  \vspace{3mm}
  \hfill \textit{Yoda (Star Wars)} \\
  \vspace{1cm}
  \pause ținem cont de ce zice lumea că e bine (presiune socială) \\
  \pause ne e rușine cu defectele noastre \\
  \pause le ținem ascunse \\
  \pause le folosim ca \textit{self-deprecation} (generează milă, nu compasiune) \\
  \pause \textbf{nu ne folosim de defectele noastre}
\end{frame}

\begin{frame}{Să analizăm}
  \centering
  \Large
  \textit{You can't do the right things with the wrong people.} \\
  \vspace{1cm}
  \pause \textbf{Avem nevoie de anti-eroi.}
\end{frame}

\begin{frame}{Explorer Reporting}
  \centering
  \Large
  \pause actorii vor să joace personaje negative, complexe \\
  \vspace{1cm}
  \pause \textit{Embrace your inner anti-hero} \\
  \vspace{3mm}
  \hfill \textit{Trailer for Venom (2018)}
\end{frame}

\begin{frame}{Călătoria}
  \centering
  \Large
  \pause Luke Skywalker \\
  \pause Shadow Walk (Starcraft) \\
  \pause Iisus \\
  \pause stare into the abyss \\
  \pause descoperi defecte, limite, unghere psihologice ascunse: \textbf{anti-eroul}
\end{frame}

\begin{frame}{Ce urmărești}
  \centering
  \Large
  \pause autenticitate \\
  \pause congruență \\
  \pause înțelegere \\
  \pause echilibru \\
  \pause forță și disciplină
\end{frame}

\begin{frame}{Cum e bine să fii}
  \centering
  \Large
  \pause sensibil vs detașat \\
  \pause toxic vs tonic \\
  \pause într-un fel congruent \\
  \pause nu există rețetă, există descoperire / călătorie \\
  \pause nu poți fi tot, nu poți face totul \\
  \pause \textbf{complementaritate}: fiecare anti-erou cu plusurile/minusurile sale
\end{frame}

\begin{frame}{Become the anti-hero: Defecte, slăbiciuni}
  \centering
  \Large
  \pause explorează \\
  \pause conștientizează \\
  \pause temperează, corectează, acceptă \\
  \pause \textbf{folosește}
\end{frame}

\begin{frame}{În final}
  \centering
  \LARGE
  \pause{Don't be good. Do good.}
\end{frame}

\end{document}
