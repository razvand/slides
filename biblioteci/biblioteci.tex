\documentclass{beamer}

\usepackage[utf8x]{inputenc}		% diacritice
\usepackage[romanian]{babel}
\usepackage{color}			% highlight
\usepackage{alltt}			% highlight

% highlight; comment this out in case you don't input code source files
%\usepackage{code/highlight}		% highlight

\usepackage{hyperref}			% folosiți \url{http://...}
					% sau \href{http://...}{Nume Link}
\usepackage{verbatim}

% Show contents at every section beginning. Ripped off from manual.
\AtBeginSection[] % Do nothing for \section*
{
  \begin{frame}<beamer>
    \frametitle{Outline}
  \tableofcontents[currentsection]
    \end{frame}
}


\mode<presentation>
{ \usetheme{Berlin} }

% Încărcăm simbolurilor Unicode românești în titlu și primele pagini
\PrerenderUnicode{aâîțșĂÎÂȚȘ}

\title[Biblioteci]{Biblioteci. Gestiunea bibliotecilor}
\institute{Întâlnirile lunare RLUG -- Iulie 2011}
\author[Răzvan Deaconescu]{Răzvan Deaconescu \\
    razvan@rosedu.org}
\date{14 iulie 2011}

\begin{document}

\frame{\titlepage}

\section{Biblioteci}

\begin{frame}{Fișiere obiect}
  \begin{itemize}
    \item fișiere binare
    \item cod compilat și asamblat
    \item format ELF, COFF, PE
  \end{itemize}
\end{frame}

\begin{frame}{Biblioteci}
  \begin{itemize}
    \item colecție de fișiere obiect
    \item funcționalitățile oferite de fișierele obiect sunt disponibile în
    bibliotecă
    \item link-area unei biblioteci pentru obținerea unui executabil
  \end{itemize}
\end{frame}

\begin{frame}{De ce biblioteci?}
  \begin{itemize}
    \item modularizare, reutilizare
    \item \textit{reinventing the wheel}
  \end{itemize}
\end{frame}

\begin{frame}{Biblioteci statice}
  \begin{itemize}
    \item colecție ``dumb'' de fișiere obiect
    \item linkarea unei biblioteci statice înseamnă adăugarea de cod
    executabil
    \item se adaugă întreg codul modulului obiect din care provine funcția
    \item \texttt{ar rc libmylib.a a.o b.o c.o}
  \end{itemize}
\end{frame}

\begin{frame}{Biblioteci dinamice}
  \begin{itemize}
    \item fișier cu format specializat în care sunt colectate fișierele obiect
    \item deține informații suplimentare despre obiecte, simboluri, funcții
    \item linkarea înseamnă doar marcarea unor referințe către bibliocă
    \item încărcarea codului din cadrul bibliotecii va fi realizată ulterior
    \item \texttt{.so} pe Unix, \texttt{.dll} pe Windows
    \item \texttt{gcc -shared -o libmylib.so a.o b.o c.o}
    \item fișierele obiect compilate cu \texttt{-fPIC}
  \end{itemize}
\end{frame}

\begin{frame}{Biblioteci statice vs. dinamice}
  \begin{itemize}
    \item statice
      \begin{itemize}
        \item cod portabil (independent de bibliotecile sistemului)
        \item nu există overhead la load-time sau run-time
      \end{itemize}
    \item dinamice
      \begin{itemize}
        \item dimensiune mică a executabilului
        \item eficiență în folosirea memoriei -- bibliotecile sunt
        ``partajate'' de mai multe procese
      \end{itemize}
  \end{itemize}
\end{frame}

\begin{frame}{Biblioteci vs. headere}
  \begin{itemize}
    \item bibliotecile sunt colecții de fișiere obiect
    \item headerele sunt fișiere cu extensia \texttt{.h} ce conțin
      \begin{itemize}
        \item declarații de funcții
        \item macrodefiniții
        \item definiții de structuri și tipuri de date
      \end{itemize}
    \item un header este inclus de un fișier C sau alt fișier header
    \item o bibliotecă este linkată la un executabil
    \item un header este scris de programator
    \item o bibliotecă este obținută prin ``comasarea'' fișierelor obiect
    \item header -- preprocesare
    \item bibliotecă -- linking
  \end{itemize}
\end{frame}

\begin{frame}{Analiza unei biblioteci}
  \begin{itemize}
    \item ldd -- listarea dependențelor pentru bibliotei dinamice
    \item nm -- listarea simbolurilor
    \item readelf -- citește fișierele ELF
  \end{itemize}
\end{frame}

\section{Programatic}

\begin{frame}{Linker-ul}
  \begin{itemize}
    \item rezolvarea simbolurilor și unirea modulelor obiect
    \item un simbol: o variabilă, un nume de funcție
    \item la compilare, simbolurile folosite în modul, dar nedefinite, sunt
    marcate special (undefined)
    \item rezolvare = descoperirea modulului în care este definit simbolurile
    \item linker-ul este folosit pentru a obține executabile și biblioteci
    dinamice
    \item \texttt{LD\_FLAGS} -- flag-uri de linker
      \begin{itemize}
        \item \texttt{-L.} -- la linkare sunt căutate bibliotecile și în
        directorul curent
      \end{itemize}
    \item \texttt{LD\_LIBS} -- biblioteci linkate
      \begin{itemize}
        \item \texttt{-ltorrent}
      \end{itemize}
  \end{itemize}
\end{frame}

\begin{frame}{Loader-ul}
  \begin{itemize}
    \item încarcă programul în execuție și începe rularea procesului
    \item load time = la execuție
    \item loaderul are cunoștință de formatul executabilului
    \item traduce zonele din fișierul executabil în zone de memorie
    \item apelat prin intermediul \texttt{execve(2)}
  \end{itemize}
\end{frame}

\begin{frame}{Static linking}
  \begin{itemize}
    \item folosit la legarea modulelor obiect și a bibliotecilor statice
    \item tot codul necesar este ``colectat'' în executabil
    \item executabilul poate fi portat pe un sistem ce nu deține bibliotecile
    folosite
    \item se poate folosi opțiunea \texttt{-static} la gcc
  \end{itemize}
\end{frame}

\begin{frame}{Dynamic linking}
  \begin{itemize}
    \item folosit la legarea bibliotecilor dinamice
    \item se adaugă puțin cod în executabil
    \item aducerea codului necesar în memorie se realizează mai târziu
  \end{itemize}
\end{frame}

\begin{frame}{Load time dynamic linking}
  \begin{itemize}
    \item codul necesar din cadrul bibliotecii este adus în memorie la load
    time (lansarea în execuție)
    \item se ocupă loaderul
    \item loaderul trebuie să știe unde să caute
      \begin{itemize}
        \item opțiunea \texttt{-L.} este folosită la \textbf{link time} nu la
        \textbf{load time}
      \end{itemize}
    \item se lansează procesul și se adaugă codul bibliotecii
    \item dacă biblioteca există în memorie (încărcată de alt proces), atunci
    este referită și partajată
  \end{itemize}
\end{frame}

\begin{frame}{Run time dynamic linking}
  \begin{itemize}
    \item codul este adus în memorie la run time (în momentul în care procesul
    se execută)
    \item similaritate cu \texttt{malloc}
    \item dezvoltatorul precizează numele bibliotecii; căutarea se face
    similar ca la load time dynamic linking
    \item \texttt{dlopen} \& friends
    \item \texttt{LoadLibrary} \& friends
    \item De ce?
      \begin{itemize}
        \item plugins
        \item hooking (injection)
      \end{itemize}
  \end{itemize}
\end{frame}

\begin{frame}{LD\_PRELOAD}
  \begin{itemize}
    \item nume de fișiere de tip bibliotecă ce sunt încărcate înaintea altora
      \begin{itemize}
        \item \textbf{nu} căi către directoare cu biblioteci (cum se întâmplă
        la \texttt{LD\_LIBRARY\_PATH}
      \end{itemize}
    \item scenariu tipic -- hooking
      \begin{itemize}
        \item se creează un modul ce implementează \texttt{malloc}
        \item se creează o bibliotecă ce conține modulul
        \item se folosește \texttt{LD\_PRELOAD}
        \item la un apel malloc se apelează biblioteca proprie
        \item folosind \texttt{dlopen} \& friends se apelează \texttt{malloc}
        din biblioteca standard C
      \end{itemize}
  \end{itemize}
\end{frame}

\begin{frame}{LD\_DEBUG}
  \begin{itemize}
    \item depanarea interacțiunii cu biblioteca
    \item \texttt{export LD\_DEBUG=help}
    \item \texttt{export LD\_DEBUG=files}
  \end{itemize}
\end{frame}

\section{Gestiunea bibliotecilor}

\begin{frame}{Biblioteci statice}
  \begin{itemize}
    \item se păstrează în directoare standard (\texttt{/usr/lib},
    \texttt{/lib})
    \item altfel, se precizează la link time calea către bibliotecă
    (\texttt{LD\_FLAGS}, \texttt{-L.})
    \item nu este nevoie de un management specializat
  \end{itemize}
\end{frame}

\begin{frame}{Biblioteci dinamice}
  \begin{itemize}
    \item rezolvarea referințelor se face la link time, fără a încărca bucăți
    de cod în fișierul executabil
    \item calea către biblioteci trebuie rezolvată la load time (sau run time)
    \item trebuie configurat loader-ul, care nu este apelat explicit de
    utilizator (este apelat prin intermediul \texttt{execve})
    \item \texttt{man ld.so}
  \end{itemize}
\end{frame}

\begin{frame}{What is this?}
  \begin{itemize}
    \item \texttt{ld.so}
    \item \texttt{ld-linux.so}
    \item \texttt{/etc/ld.so.conf}
    \item \texttt{/etc/ld.so.conf.d/}
    \item \texttt{/etc/ld.so.cache}
    \item \texttt{/etc/ld.so.preload}
  \end{itemize}
\end{frame}

\begin{frame}{LD\_LIBRARY\_PATH}
  \begin{itemize}
    \item precizează căi suplimentare unde să fie căutate bibliotecile
    dinamice
    \item similar cu \texttt{PATH} -- separație prin caracterul ``două
    puncte''/``colon'' (\texttt{:})
    \item \texttt{LD\_LIBRARY\_PATH=. ./exec}
  \end{itemize}
\end{frame}

\begin{frame}{ldconfig}
  \begin{itemize}
    \item controlează cache-ul de de biblioteci dinamice
    \item cache-ul este stocat în fișierul \texttt{/etc/ld.so.cache}
    \item loader-ul caută bibliotecile urmând un set de pași dați (\texttt{man
    ld.so})
    \item înainte de a căuta în directoarele implicite (\texttt{/lib},
    \texttt{/usr/lib}), va căuta în cache -- viteză sporită
    \item \texttt{ldconfig} este apelat în general la instalarea de aplicații
    de sistemul de gestiune a pachetelor (\texttt{apt}, \texttt{yum})
    \item \texttt{ldconfig -p}
    \item \texttt{ldconfig -n .}
  \end{itemize}
\end{frame}

\section{Întrebări}

\begin{frame}{Resurse utile}
  \begin{itemize}
    \item \url{http://www.dwheeler.com/program-library/}
    \item MSDNAA -- Dynamic Link Libraries --
    \url{http://msdn.microsoft.com/en-us/library/ms682589(v=VS.85).aspx}
  \end{itemize}
\end{frame}

\begin{frame}{Keywords}
  \begin{columns}
    \begin{column}[l]{0.5\textwidth}
      \begin{itemize}
        \item fișiere obiect
        \item biblioteci
        \item biblioteci statice
        \item biblioteci dinamice
        \item \texttt{-fPIC}
        \item header
        \item linker
        \item loader
      \end{itemize}
    \end{column}
    \begin{column}[l]{0.5\textwidth}
      \begin{itemize}
        \item \texttt{ld.so}
        \item run-time
        \item load-time
        \item \texttt{LD\_PRELOAD}
        \item \texttt{LD\_DEBUG}
        \item \texttt{LD\_LIBRARY\_PATH}
        \item \texttt{/etc/ld.*}
        \item \texttt{ldconfig}
      \end{itemize}
    \end{column}
  \end{columns}
\end{frame}

\end{document}
