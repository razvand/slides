% vim: set tw=78 tabstop=4 shiftwidth=4 aw ai:
\documentclass{beamer}

\usepackage[utf8x]{inputenc}		% diacritice
\usepackage[romanian]{babel}
\usepackage{color}			% highlight
\usepackage{alltt}			% highlight

% highlight; comment this out in case you don't input code source files
%\usepackage{code/highlight}		% highlight

\usepackage{hyperref}			% folosiți \url{http://...}
					% sau \href{http://...}{Nume Link}
\usepackage{verbatim}

\mode<presentation>
{ \usetheme{Berlin} }

% Încărcăm simbolurilor Unicode românești în titlu și primele pagini
% Încărcăm simbolurilor Unicode românești în titlu și primele pagini.
\PrerenderUnicode{aâîțșĂÎÂȚȘ}

\title[rss2email]{rss2email}
\subtitle{Lightning talk}
\institute{Întâlnirile lunare RLUG -- Septembrie 2011}
\author[Răzvan Deaconescu]{Răzvan Deaconescu\\
      razvan@rosedu.org}
\date{8 septembrie 2011}

\begin{document}

% Slide-urile cu mai multe părți sunt marcate cu textul (cont.)
\setbeamertemplate{frametitle continuation}[from second]

\frame{\titlepage}

% Titlul unui frame se specifică fie în acolade, imediat după \begin{frame},
% fie folosind \frametitle
\begin{frame}{Contextul}
  \begin{itemize}     % Just like normal LaTeX
    \item folosesc feed-uri pentru notificări (wiki-uri, Redmine)
    \item Liferea are probleme cu HTTPS (probabil plus SNI)
    \item dat e-mail pe lista de devel de la Liferea, nu am primit răspuns
    \item pluginul de feed de la Evolution nu merge
    \item prefer să folosesc o aplicație (clientul de e-mail) pentru mai multe
    ``task-uri'': e-mail, calendaring, feed notification
  \end{itemize}
\end{frame}

\begin{frame}{Soluția: rss2email}
  \begin{itemize}
    \item convertește notificările RSS/Atom în email-uri
    \item ușor de instalat
    \item configurare rapidă
    \item îl configurezi pe un server și de acolo îți trimite mesaje
      \begin{itemize}
        \item nu e nevoie de configurare pe client și, apoi, reconfigurare, pe
        alt client
      \end{itemize}
  \end{itemize}
\end{frame}

\begin{frame}{Instalare și utilizare}
  \begin{itemize}
    \item instalare pe un server stabil
    \item \texttt{apt-get install rss2email}
    \item \texttt{r2e} (fără argumente afișează ecran de ajutor)
    \item \texttt{r2e new razvan@rosedu.org}
    \item \texttt{r2e add
    '\url{http://lif.rosedu.org/wiki/feed.php?linkto=diff\&content=diff\&mode=recent}'}
      \begin{itemize}
        \item Atenție la ampersand!
      \end{itemize}
    \item \texttt{r2e run --no-send} (verificare feed-uri)
    \item \texttt{r2e list}
    \item \texttt{r2e delete 7}
    \item intrare în crontab
      \begin{itemize}
        \item \texttt{0 * * * * /usr/bin/r2e run $>$ /dev/null 2$>$\&1}
      \end{itemize}
  \end{itemize}
\end{frame}

\begin{frame}{Configurare}
  \begin{itemize}
    \item \texttt{cp /usr/share/doc/rss2email/examples/config.py
    $\sim$/.rss2email/config.py}
    \item \texttt{DEFAULT\_FROM = "rss2email-noreply@swarm.cs.pub.ro"}
    \item \texttt{SMTP\_SERVER = "localhost:25"}
  \end{itemize}
\end{frame}

\begin{frame}{``Personalizare''}
  \begin{itemize}
    \item \texttt{DATE\_HEADER = 1}
    \item \texttt{FORCE\_FROM = 0}
    \item \texttt{BONUS\_HEADER = '\textbackslash{}nX-feed:
    rss2email-noreply@swarm.cs.pub.ro'}
      \begin{itemize}
        \item plus filtru în MDA sau clientul de e-mail
      \end{itemize}
  \end{itemize}
\end{frame}

\end{document}
