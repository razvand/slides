% vim: set tw=78 tabstop=4 shiftwidth=4 aw ai:
\documentclass{beamer}

\usepackage[utf8x]{inputenc}		% diacritice
\usepackage[romanian]{babel}
\usepackage{color}			% highlight
\usepackage{alltt}			% highlight

% highlight; comment this out in case you don't input code source files
%\usepackage{code/highlight}		% highlight

\usepackage{hyperref}			% folosiți \url{http://...}
					% sau \href{http://...}{Nume Link}
\usepackage{verbatim}

\mode<presentation>
{ \usetheme{Berlin} }

% Încărcăm simbolurilor Unicode românești în titlu și primele pagini
% Încărcăm simbolurilor Unicode românești în titlu și primele pagini.
\PrerenderUnicode{aâîțșĂÎÂȚȘ}

\title[\LaTeX]{De ce \LaTeX?}
\subtitle{Rival Ideas}
\institute{ROSEdu/ACS}
\author[Răzvan Deaconescu]{Răzvan Deaconescu\\
      razvan.deaconescu@cs.pub.ro}
\date{9 octombrie 2011}

\begin{document}

% Slide-urile cu mai multe părți sunt marcate cu textul (cont.)
\setbeamertemplate{frametitle continuation}[from second]

\frame{\titlepage}

\begin{frame}{De ce \LaTeX?}
  \begin{itemize}
    \pause \item folosești editorul preferat
    \pause \item repository
    \pause \item text (grep, sed etc.), automatizare/scripting
    \pause \item scalabil (include) (500 pagini în \LaTeX -- ușor)
    \pause \item low-memory footprint
    \pause \item portabil
    \pause \item moca
    \pause \item typesetting (formule matematice)
    \pause \item extensibil (module diverse) -- \url{http://www.ctan.org}
    \pause \item documentație pe tot Internet-ul
    \pause \item Donald Knuth
    \pause \item professional crowd (the cool stuff)
    \pause \item ``why \LaTeX'' pe Google
  \end{itemize}
\end{frame}

\begin{frame}{Q\&A}
  \begin{itemize}
    \item întrebări
    \item curiozități
    \item laude
    \item hip-hip for the coolness factor
  \end{itemize}
\end{frame}

\end{document}
