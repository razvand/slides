% vim: set tw=78 tabstop=4 shiftwidth=4 aw ai:
\documentclass{beamer}

\usepackage[utf8x]{inputenc}		% diacritice
\usepackage[romanian]{babel}
\usepackage{color}			% highlight
\usepackage{alltt}			% highlight

% highlight; comment this out in case you don't input code source files
%\usepackage{code/highlight}		% highlight

\usepackage{hyperref}			% folosiți \url{http://...}
					% sau \href{http://...}{Nume Link}
\usepackage{verbatim}

\mode<presentation>
{ \usetheme{Berlin} }

% Încărcăm simbolurilor Unicode românești în titlu și primele pagini
\PreloadUnicodePage{200}

\title[Gitosis/Gitolite]{Gitosis/Gitolite}
\subtitle{Utilizare și internals}
\institute{Întâlnirile lunare RLUG -- Septembrie 2010}
\author[Răzvan Deaconescu]{Răzvan Deaconescu\\
	razvan@rosedu.org}
\date{16 septembrie 2010}

\begin{document}

% Slide-urile cu mai multe părți sunt marcate cu textul (cont.)
\setbeamertemplate{frametitle continuation}[from second]

% Arătăm numărul frame-ului
%\setbeamertemplate{footline}[frame number]

\frame{\titlepage}

\frame{\tableofcontents}

% NB: Secțiunile nu sunt marcate vizual, ci doar apar în cuprins
\section{Git}

% Titlul unui frame se specifică fie în acolade, imediat după \begin{frame},
% fie folosind \frametitle
\begin{frame}{Git}
	\begin{itemize}		% Just like normal LaTeX
		\item aaa
		\item aaa
	\end{itemize}
\end{frame}

\begin{frame}{Folosire Git}
	\begin{itemize}
		\item aa
		\item bbb
	\end{itemize}
\end{frame}

\begin{frame}{Internele Git}
	\begin{itemize}
		\item aa
		\item bbb
	\end{itemize}
\end{frame}

\begin{frame}{Internele Git -- objects}
	\begin{itemize}
		\item aa
		\item bbb
	\end{itemize}
\end{frame}

\begin{frame}{Internele Git -- refs}
	\begin{itemize}
		\item aa
		\item bbb
	\end{itemize}
\end{frame}

\begin{frame}{Internele Git -- packs}
	\begin{itemize}
		\item aa
		\item bbb
	\end{itemize}
\end{frame}

\begin{frame}{URL-uri Git}
	\begin{itemize}
		\item aa
		\item bbb
	\end{itemize}
\end{frame}

\section{Gitosis}

\frame{\tableofcontents[currentsection]}

\begin{frame}{Gitosis}
	\begin{itemize}
		\item aa
		\item bbb
	\end{itemize}
\end{frame}

\begin{frame}{Instalare}
	\begin{itemize}
		\item aa
		\item bbb
	\end{itemize}
\end{frame}

\begin{frame}{Utilizare}
	\begin{itemize}
		\item aa
		\item bbb
	\end{itemize}
\end{frame}

\begin{frame}{Internele Gitosis}
	\begin{itemize}
		\item aa
		\item bbb
	\end{itemize}
\end{frame}

\section{Gitolite}

\frame{\tableofcontents[currentsection]}

\begin{frame}{Gitolite}
	\begin{itemize}
		\item aa
		\item bbb
	\end{itemize}
\end{frame}

\begin{frame}{Instalare}
	\begin{itemize}
		\item aa
		\item bbb
	\end{itemize}
\end{frame}

\begin{frame}{Utilizare}
	\begin{itemize}
		\item aa
		\item bbb
	\end{itemize}
\end{frame}

\begin{frame}{Internele Gitolite}
	\begin{itemize}
		\item aa
		\item bbb
	\end{itemize}
\end{frame}

\section{Întrebări}

\begin{frame}{Resurse utile}
	\begin{itemize}
		\item \url{http://git-scm.com/}
		\item \url{http://progit.org/}
		\item \url{http://github.com/res0nat0r/gitosis}
		\item \url{http://github.com/sitaramc/gitolite}
	\end{itemize}
\end{frame}

\frame{\tableofcontents[currentsection]}

\end{document}
