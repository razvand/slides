\documentclass[handout]{beamer}

% Romanian Language support
\usepackage{ucs}
\usepackage{alltt}
\usepackage[utf8x]{inputenc}
\PrerenderUnicode{aâîțșĂÎÂȚȘ}
\usepackage[english,romanian]{babel}

\usepackage{hyperref}   % use \url{http://$URL} or \href{http://$URL}{Name}
\usepackage{verbatim}
\usepackage{underscore} % underscores need not be escaped
\usepackage{booktabs}   % nice looking tables
\usepackage{array}      % column size options in tables
\usepackage[normalem]{ulem}       % for striketrough text

\mode<presentation>
%{ \usetheme{Berlin} }

% Disable useless navigation symbols.
\setbeamertemplate{navigation symbols}{}

\title[Lucruri care contează]{Lucruri care contează}
\institute{Excursie PRECIS 708 și prietenii}
\author[Răzvan Deaconescu]{Răzvan Deaconescu \\
razvan.deaconescu@cs.pub.ro}
\date{17 februarie 2018}

\begin{document}

\frame{\titlepage}

\begin{frame}{Atribute ale celor cu care lucrăm, colaborăm}
  \begin{enumerate}
    \pause \item cele care sunt utile, e bine să le ai, dar nu sunt diferențiatori
    \pause \item cele care \textbf{contează}
  \end{enumerate}
\end{frame}

\begin{frame}{Cele utile care nu contează atât de mult}
  \begin{figure}
    \centering
    \includegraphics[width=\textwidth]{img/soundsgood.jpg}
  \end{figure}
  \begin{center}
    \scriptsize
    \url{http://knowyourmeme.com/memes/sounds-good-doesnt-work}
  \end{center}
\end{frame}

\begin{frame}{Care sunt cele care contează}
  \centering
  \pause \Large{Gândiți-vă la una.}
\end{frame}

\begin{frame}{Focus}
  \begin{itemize}
    \pause \item concentrare
    \pause \item do one thing, do \textbf{that} thing
    \pause \item rezistență/imunitate la ,,deranj'' extern
    \pause \item bun pentru productivitate
    \pause \item Lucian
    \pause \item Tavi
  \end{itemize}
\end{frame}

\begin{frame}{Empatie}
  \begin{itemize}
    \pause \item ascultă (nu doar auzi)
    \pause \item \textbf{nu} judeca
    \pause \item înțelege, fii acolo, \textbf{nu} oferi soluții
    \pause \item \textbf{nu} reproșa
    \pause \item bun pentru atmosfera echipei
    \pause \item Mihai Țigănuș
    \pause \item Răzvan Nițu
  \end{itemize}
\end{frame}

\begin{frame}{Compulsivitate}
  \begin{itemize}
    \pause \item faci pentru că simți nevoia să faci, nu pentru că a zis cineva
    \pause \item nu te poți abține, e dincolo de tine
    \pause \item \textbf{I have to do it}
    \pause \item bun pentru ca lucrurile să fie făcute, fără nevoie de remindere
    \pause \item Alexandra
    \pause \item Mihai Carabaș
  \end{itemize}
\end{frame}

\begin{frame}{Going the Extra Mile}
  \begin{itemize}
    \pause \item disponibilitate
    \pause \item lucrurile sunt făcute \textbf{foarte bine}
    \pause \item apar lucruri neprevăzute
    \pause \item bun pentru a duce treaba la bun sfârșit indiferent de împrejurări
    \pause \item Alex Căciulescu
    \pause \item compulsivii
  \end{itemize}
\end{frame}

\begin{frame}{Autonomie}
  \begin{itemize}
    \pause \item self-driven, self-directed
    \pause \item își asumă, nu se plânge
    \pause \item motivație internă
    \pause \item bun pentru delegare
    \pause \item Cristian Condurache
  \end{itemize}
\end{frame}

\begin{frame}{Enjoys the Ride}
  \begin{itemize}
    \pause \item îi place ce face
    \pause \item intră în \textit{flow}, automotivat
    \pause \item se bucură de rezultate mici
    \pause \item bun pentru delegare, fără nevoie de ,,motivare''
    \pause \item Edi
  \end{itemize}
\end{frame}

\begin{frame}{Direct}
  \begin{itemize}
    \pause \item \textit{say what I need to hear, not what I want to hear}
    \pause \item \textbf{no} bullshit
    \pause \item autentic
    \pause \item bun pentru a raporta și rezolva rapid problemele
    \pause \item Mihai Chiroiu
    \pause \item Lucian, Mihai Carabaș
    \pause \item Larisa
  \end{itemize}
\end{frame}

\begin{frame}{Răbdare}
  \begin{itemize}
    \pause \item lucrurile nu se întâmplă \textbf{acum}
    \pause \item e nevoie de timp, trial and error
    \pause \item iterații
    \pause \item formă de realism
    \pause \item bun pentru a face lucrurile \textbf{mai bine}
    \pause \item Nicolae Țăpuș
    \pause \item Răzvan Nițu
  \end{itemize}
\end{frame}

\begin{frame}{Disciplină}
  \begin{itemize}
    \pause \item rigoare internă
    \pause \item faci ce trebuie și nu faci ce nu trebuie
    \pause \item gestiunea timpului și atenției
    \pause \item bun pentru stabilitate, consecvență
    \pause \item Vali Ghiță
  \end{itemize}
\end{frame}

\begin{frame}{Cum faci ca cei din jur să aibă atributele astea?}
  \centering
  \pause \Large{Ajutătoare: Cum faci pe cineva să-i placă de tine?}\\
  \vspace{0.5cm}
  \pause \Large{pe scurt: Nu poți}\\
  \vspace{0.5cm}
  \pause \Large{Some people don't like you.}\\
\end{frame}

\begin{frame}{Wisdom Moment \#73}
  \centering
  \pause \Large{E mai ușor să \sout{schimbi} înlocuiești un om decât să \sout{schimbi} transformi un om.}
\end{frame}

\begin{frame}{Wisdom Moment \#74}
  \centering
  \pause \Large{Oamenii trebuie să accepte sau, mai bine, să își dorească să se schimbe sau se îmbunătățească.}\\
  \vspace{0.5cm}
  \pause tranzacțional vs. internalizat
\end{frame}

\begin{frame}{Nu e vorba doar de cei cu care lucrăm}
  \begin{itemize}
    \pause \item de cei din jurul nostru
    \pause \item prieteni, apropiați, familie
  \end{itemize}
\end{frame}

\begin{frame}{Popa Tanda}
  \begin{itemize}
    \pause \item \url{https://ro.wikisource.org/wiki/Popa\_Tanda}
    \pause \item popa din Sărăceni
    \pause \item trei faze
    \pause \item sfat: ,, Așteptați! grăi el. Dacă nu veniți voi la mine, mă duc eu la voi! Și apoi porni popa la colindă. Cât e ziua de mare, gura lui nu se mai oprea. Unde prindea oamenii, acolo îi ținea la sfaturi. La câmp dai de popă; la deal dă popa de tine; mergi la vale, te întâlnești cu popa; intri-n pădure, tot pe popa îl afli. Popa la biserică, popa la mort, popa la nuntă, popa la vecin: trebuie să fugi din sat dacă voiești să scapi de popa. Și unde te prinde te omoară cu sfatul.''
    \pause \item batjocură
    \pause \item ocară
    \pause \item nu iese nimic
    \pause \item se apucă de treabă singur, devine un model, schimbă satul
  \end{itemize}
\end{frame}

\begin{frame}{Influență}
  \begin{itemize}
    \pause \item poți afecta calități ale celor din jur
    \pause \item nu există rețete
    \pause \item contează ce faci, nu ce zici
    \pause \item fiecare reprezintă un model
  \end{itemize}
\end{frame}

\begin{frame}{Dacă vrei ceva la cei din jur \ldots}
  \centering
  \pause \Large{\ldots{} trebuie să demonstrezi tu asta.}\\
  \vspace{0.5cm}
  \pause mirroring, proiecție
\end{frame}

\begin{frame}{Watch this}
  \begin{itemize}
    \item \url{https://www.youtube.com/watch?v=Td0pUwrBWjc}
    \pause \item \textit{I am my scars!}
    \pause \item \textit{Only we can save ourselves.}
  \end{itemize}
\end{frame}

\begin{frame}{Cristina și Mihai}
  \begin{itemize}
    \item Cristina e nebună.
    \item Mihai are dinții strâmbi.
  \end{itemize}
\end{frame}

\begin{frame}{Cum dobândim abilități? Cum devenim mai buni?}
  \begin{itemize}
    \pause \item se poate antrena
    \pause \item nu există rețete
    \pause \item unele lucruri sunt naturale și pot fi îmbunătățite
    \pause \item alte lucruri trebuie educate și apoi educate și apoi educate
    \pause \item nu e ușor
  \end{itemize}
\end{frame}

\begin{frame}{Sfaturi/idei}
  \begin{itemize}
    \pause \item slow down
      \begin{itemize}
        \pause \item Breathe, just breathe. (Luke Skywalker, ,,The Last Jedi'')
      \end{itemize}
    \pause \item observați
    \pause \item daydreaming
    \pause \item mindset de răbdare, nu vă grăbiți, nu se termină lumea
    \pause \item small talk despre subiecte \textbf{altele} decât vreme, trafic, ce faci de revelion?
    \pause \item depănați
  \end{itemize}
\end{frame}

\begin{frame}{Concluzie}
  \begin{itemize}
    \pause \item lucrurile care contează sunt cele care fac diferența
    \pause \item sunt ceea ce căutăm la cei din jur (colaboratori, apropiați)
    \pause \item e mai ușor să înlocuim pe cineva decât să transformăm pe cineva
    \pause \item modul de influență ține de noi, de ce facem, de ce model suntem; vorbitul e doar o formă de exprimare
    \pause \item noi trebuie să ne antrenăm pe noi, să ne îmbunătățim înainte de a cere asta de la ceilalți
  \end{itemize}
\end{frame}

\begin{frame}{În final}
  \pause
  \centering
  \LARGE{\textit{Waste no more time arguing what a good man should be. Be one.}} \\
  \vspace{3mm}
  \hfill \normalsize{\textit{Marcus Aurelius}} \\
\end{frame}

\end{document}
