\documentclass{beamer}

% Romanian Language support
\usepackage{ucs}
\usepackage[utf8x]{inputenc}
\PrerenderUnicode{aâîțșĂÎÂȚȘ}
\usepackage[english,romanian]{babel}

\usepackage{hyperref}   % use \url{http://$URL} or \href{http://$URL}{Name}
\usepackage{verbatim}
\usepackage{underscore} % underscores need not be escaped
\usepackage{booktabs}   % nice looking tables
\usepackage{array}      % column size options in tables
\usepackage[normalem]{ulem}       % for striketrough text

\mode<presentation>
%{ \usetheme{Berlin} }

% Disable useless navigation symbols.
\setbeamertemplate{navigation symbols}{}

\title[Despre dezvăț]{Despre dezvăț: Uneori strici ca să construiești}
\institute{Info Educație 2016 (Gălăciuc, Vrancea)}
\author[Răzvan Deaconescu]{Răzvan Deaconescu \\
razvan.deaconescu@cs.pub.ro}
\date{4 august 2016}

\begin{document}

\frame{\titlepage}

\begin{frame}{Good guys vs. bad guys}
  good guys finish first
  bad guys finish last
  bad guys get the girl
  încredere în sine
  Tony Montana (Scarface)
\end{frame}

\begin{frame}{Cuminte vs. băgăcios}
  stai în banca ta
  bagă-te în față, dă din coate
  Winston Churchill
\end{frame}

\begin{frame}{Cum învățăm?}
  de la cei din jur
  din experiență
  din societate
  din ce vedem, auzim, ascultăm
  din ce citim
  de pe Internet
\end{frame}

\begin{frame}{Problema Internet-ului}
  multă informație
  informația nu este filtrată
  îți dă soluția la problemă, nu cum ajungi la ea
  Google/SO considered harmful
  Computers are useless
\end{frame}

\begin{frame}{Imperfecțiuni}
  informații incomplete
  obieceiuri ineficiente
  presiune socială: așa e bine, așa se întâmplă
  poveste cu cine nu putea să se strâmbe
\end{frame}

\begin{frame}{Absolutizare / Depinde}
  fii tu însuți
  dacă vrei poți
  cu pasiune rezolvi orice
  trebuie să te comporți frumos cu cei din jur
  toți politicienii sunt corupți
  e nasol în România
\end{frame}

\begin{frame}{De ce se întâmplă?}
  lene
  lipsă de răbdare
  neconștientizarea alternativei
  influență mare a celor din jur
  absența gândirii critice
  bias-uri: confirmation bias
  instinctul de turmă, ,,compliance''
  bulă socială
  experimentul cu maimuțele
  \url{http://www.throwcase.com/2014/12/21/that-five-monkeys-and-a-banana-story-is-rubbish/}
\end{frame}

\begin{frame}{De ce e rău că se întâmplă?}
  împrăștii dezinformare, cultivre de ignoranță
  nu îți atingi potențialul, te auto-limitezi
  te rigidizezi, nu te adaptezi
  lumea este în mișcare rapidă, e dinamică
\end{frame}

\begin{frame}{Atenție la vârstă}
  Marin Preda
  Anakin Skywalker
\end{frame}

\begin{frame}{Despre dezvăț}
  renunțat la obiceiuri, comportamente, gânduri, moduri de abordare
  alterat obiceiuri, comportamente, gânduri, moduri de abordare
  stricat și construit
  Pink Floyd: Another Brick in the Wall
\end{frame}

\begin{frame}{Exemple și discuție}
  Răzvan
\end{frame}

\begin{frame}{Pași}
  conștientizare
  găsit soluție
  efort
  reconstruire
  menținut flexibilitate
\end{frame}

\begin{frame}{Feedback}
  DAWA
  oferă, primește, cere, folosește, reiterează
  poză
\end{frame}

\begin{frame}{Recomandări}
  Gândiți-vă la probleme, nu doar la soluții.
  Faceți jocuri de logică, gândire laterală.
  Interacționați cu oameni diferiți.
  Jocuri de logică și gândire laterală.
  Învățați programare funcțională.
  Faceți lucrurile și altfel.
  Meditați, analizați-vă.
  Enjoy!
\end{frame}

\begin{frame}{În loc de final}
  Ancora Imparo (Michelangelo, 87 de ani)
\end{frame}

\end{document}
