% vim: set tw=78 tabstop=4 shiftwidth=4 aw ai:
\documentclass{beamer}

\usepackage[utf8x]{inputenc}		% diacritice
\usepackage[romanian]{babel}
\usepackage{color}			% highlight
\usepackage{alltt}			% highlight

% highlight; comment this out in case you don't input code source files
%\usepackage{code/highlight}		% highlight

\usepackage{hyperref}			% folosiți \url{http://...}
					% sau \href{http://...}{Nume Link}
\usepackage{verbatim}

\mode<presentation>
{ \usetheme{Berlin} }

% Încărcăm simbolurilor Unicode românești în titlu și primele pagini
\PreloadUnicodePage{200}

\title[Cluster SSH / Parallel SSH]{Cluster SSH / Parallel SSH}
\subtitle{Lightning talk}
\institute{Întâlnirile lunare RLUG -- Septembrie 2010}
\author[Răzvan Deaconescu]{Răzvan Deaconescu\\
	razvan@rosedu.org}
\date{16 septembrie 2010}

\begin{document}

% Slide-urile cu mai multe părți sunt marcate cu textul (cont.)
\setbeamertemplate{frametitle continuation}[from second]

% Arătăm numărul frame-ului
%\setbeamertemplate{footline}[frame number]

\frame{\titlepage}

% NB: Secțiunile nu sunt marcate vizual, ci doar apar în cuprins
\section{Cluster SSH}

\frame{\tableofcontents[currentsection]}

% Titlul unui frame se specifică fie în acolade, imediat după \begin{frame},
% fie folosind \frametitle
\begin{frame}{Cluster SSH}
	\begin{itemize}		% Just like normal LaTeX
		\item conexiuni SSH multiple
		\item câte o fereastră de terminal pe conexiune
		\item o ``fereastră specială'' de comandă
			\begin{itemize}
				\item comenzile se execută pe toate sistemele
				\item comenzile în ferestrele obișnuite se execută pe fiecare
				sistem
			\end{itemize}
		\item execuția aceleiași comenzi pe mai multe sisteme
		\item \texttt{apt-get install clusterssh}
	\end{itemize}
\end{frame}

\begin{frame}{Configurare}
	\begin{itemize}
		\item tag-based
		\item \texttt{/etc/clusters}
			\begin{itemize}
				\item TODO
			\end{itemize}
		\item \texttt{/etc/csshrc}, \texttt{\$HOME/.csshrc}
			\begin{itemize}
				\item TODO
			\end{itemize}
		\item Cum fac eu?
			\begin{itemize}
				\item TODO
			\end{itemize}
		\item man cssh
	\end{itemize}
\end{frame}

\begin{frame}{Utilizare}
	\begin{itemize}
		\item cssh s1 s2 s3
		\item cssh tag
	\end{itemize}
\end{frame}

\section{Parallel SSH}

\frame{\tableofcontents[currentsection]}

\begin{frame}{Parallel SSH}
	\begin{itemize}
		\item TODO
	\end{itemize}
\end{frame}

\begin{frame}{Configurare}
	\begin{itemize}
		\item TODO
	\end{itemize}
\end{frame}

\begin{frame}{Utilizare}
	\begin{itemize}
		\item TODO
	\end{itemize}
\end{frame}

\section{Întrebări}

\frame{\tableofcontents[currentsection]}

\begin{frame}{Resurse utile}
	\begin{itemize}
		\item
		\url{http://www.debianadmin.com/ssh-on-multiple-servers-using-cluster-ssh.html}
		\item \url{http://sourceforge.net/projects/clusterssh/}
		\item \url{http://code.google.com/p/parallel-ssh/}
		\item \url{http://www.linux.com/archive/articles/151340}
		\item
		\url{http://serverfault.com/questions/17931/what-is-a-good-modern-parallel-ssh-tool}
	\end{itemize}
\end{frame}

\end{document}
