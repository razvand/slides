\documentclass{simple}

\usepackage{code/highlight}

\title[Cluster SSH / Parallel SSH]{Cluster SSH / Parallel SSH}
\subtitle{Lightning talk}
\institute{Întâlnirile lunare RLUG -- Septembrie 2010}
\author[Răzvan Deaconescu]{Răzvan Deaconescu\\
	razvan@rosedu.org}
\date{16 septembrie 2010}

\begin{document}

\frame{\titlepage}

% NB: Secțiunile nu sunt marcate vizual, ci doar apar în cuprins
\section{Cluster SSH}

\frame{\tableofcontents[currentsection]}

% Titlul unui frame se specifică fie în acolade, imediat după \begin{frame},
% fie folosind \frametitle
\begin{frame}{Cluster SSH}
	\begin{itemize}		% Just like normal LaTeX
		\item conexiuni SSH multiple
		\item câte o fereastră de terminal pe conexiune
		\item o ``fereastră specială'' de comandă
			\begin{itemize}
				\item comenzile se execută pe toate sistemele
				\item comenzile în ferestrele obișnuite se execută pe fiecare
				sistem
			\end{itemize}
		\item execuția aceleiași comenzi pe mai multe sisteme
		\item \texttt{apt-get install clusterssh}
	\end{itemize}
\end{frame}

\begin{frame}{Configurare}
	\begin{itemize}
		\item tag-based
		\item \texttt{/etc/clusters}
			\begin{itemize}
				\item fișier global de definire a clusterelor
				\item \texttt{$<$tag$>$ user1@host1:port1 user2@host2:port2}
			\end{itemize}
		\item \texttt{/etc/csshrc}, \texttt{\$HOME/.csshrc}
			\begin{itemize}
				\item opțiuni de configurare
				\item precizarea unui fișier nou pentru clustere
			\end{itemize}
		\item Cum fac eu?
			\begin{itemize}
				\item \texttt{extra\_cluster\_file=\$HOME/.clusters}
				\item în \texttt{\$HOME/.clusters} definesc clusterele
			\end{itemize}
		\item \texttt{man cssh}
	\end{itemize}
\end{frame}

\begin{frame}{Utilizare}
	\begin{itemize}
		\item cssh s1 s2 s3
		\item cssh tag
	\end{itemize}
\end{frame}

\section{Parallel SSH}

\frame{\tableofcontents[currentsection]}

\begin{frame}{Parallel SSH}
	\begin{itemize}
		\item versiuni paralele ale
			\begin{itemize}
				\item ssh $\rightarrow$ parallel-ssh
				\item scp $\rightarrow$ parallel-scp, parallel-slurp
				\item rsync $\rightarrow$ parallel-rsync
				\item kill $\rightarrow$ parallel-nuke
			\end{itemize}
		\item \texttt{apt-get install parallel-ssh}
	\end{itemize}
\end{frame}

\begin{frame}{Configurare}
	\begin{itemize}
		\item un fișier de configurare în care sunt precizate sistemele
		\item format \texttt{[username@]hostname[:port]}
	\end{itemize}
\end{frame}

\begin{frame}{Utilizare}
	\input{code/commands}
\end{frame}

\section{Întrebări}

\frame{\tableofcontents[currentsection]}

\begin{frame}{Resurse utile}
	\begin{itemize}
		\item
		\url{http://www.debianadmin.com/ssh-on-multiple-servers-using-cluster-ssh.html}
		\item \url{http://sourceforge.net/projects/clusterssh/}
		\item \url{http://code.google.com/p/parallel-ssh/}
		\item \url{http://www.linux.com/archive/articles/151340}
		\item
		\url{http://serverfault.com/questions/17931/what-is-a-good-modern-parallel-ssh-tool}
	\end{itemize}
\end{frame}

\end{document}
