\documentclass{simple}

\usepackage{subcaption}

\title[Extreme, absolutisme și relativisme]{Extreme, absolutisme și relativisme}
\institute{InfoEducație 2020 (online)}
\author[Răzvan Deaconescu]{Răzvan Deaconescu \\
razvan.deaconescu@cs.pub.ro}
\date{30 iulie 2020}

\begin{document}

\frame{\titlepage}

\begin{frame}{Despre extreme (absolutisme)}
  \pause
  \textbf{Side A}: Toți oamenii sunt buni. \\
  \textbf{Tu}: Nu sunt de acord. \\
  \textbf{Side B}: Aha, deci crezi că toți oamenii sunt răi. \\
\end{frame}

\begin{frame}{Despre extreme (absolutisme) (2)}
  \pause
  \textbf{Side A}: Ești feminist(ă)? \\
  \textbf{Tu}: Nu. \\
  \textbf{Side B}: Aha, deci ești misogin(ă). \\
\end{frame}

\begin{frame}{Despre extreme (absolutisme) (3)}
  \pause
  \textbf{X}: Ce-i mai rău decât un om fără idei? \\
  \textbf{Y}: Un om cu (prea multe) idei.
\end{frame}

\begin{frame}{Despre ,,calea de mijloc''}
  \pause
  \textbf{Side A}: Kill 1,000 kittens. \\
  \textbf{Side B}: Don't kill any kittens. \\
  \textbf{Centrist}: Kill 500 kittens. \\
  \vspace{3mm}
  \pause
  \textbf{Side A}: Kill 1,000 kittens. \\
  \textbf{Side B}: Don't kill any kittens. \\
  \textbf{Centrist}: I strongly agree with Side B on this issue but that doesn't mean they're right about everything. \\
  \vspace{5mm}
  \hfill \textit{Steve Stewart-Williams (@stevestuwill)}
\end{frame}

\begin{frame}{Despre extreme (relativisme)}
  \pause
  \textbf{Tu}: Cu cine votezi? \\
  \textbf{X}: Ce contează? Toți sunt la fel. \\
  \textbf{Tu}: Dar dacă tu ai candida? \\
  \vspace{3mm}
  \pause
  \textbf{Tu}: E sănătos să nu bei și să nu fumezi. \\
  \textbf{X}: Bunicul meu a fumat și a băut ca stinsul și a murit la 86 de ani. \\
  \vspace{3mm}
  \pause
  \textit{Și-așa-i bine, și-așa-i bine, cum o dai ca dracu' vine.} \\
  \vspace{3mm}
  \pause
  Solomon și sfetnicul
\end{frame}

\begin{frame}{Răspunsuri simple}
  \begin{itemize}
    \pause
    \item X e mereu adevărat sau e mereu fals.
    \pause
    \item Adevărul e undeva la mijloc.
    \pause
    \item Totul e relativ.
  \end{itemize}
\end{frame}

\begin{frame}{Cum te poziționezi?}
  \begin{itemize}
    \pause
    \item Notele conteză / sunt importante.
    \pause
    \item Familia este importantă.
    \pause
    \item Este bine să dormi cel puțin 8 ore pe noapte.
    \pause
    \item Rasismul este o problemă globală.
    \pause
    \item Violența este condamnabilă.
    \pause
    \item Oamenii trăiesc pentru iubire.
    \pause
    \item Banii aduc fericirea.
    \pause
    \item Cel mai ușor motivezi pe cineva oferindu-i bani.
    \pause
    \item Religia este soluția.
    \pause
    \item Femeia trebuie să fie supusă bărbatului.
  \end{itemize}
\end{frame}

\begin{frame}{Răspunsuri nuanțate}
  \begin{itemize}
    \pause
    \item X e mereu adevărat.
    \pause
    \item X e de obicei adevărat, există excepții.
    \pause
    \item Depinde de situație
    \pause
    \item X e de obicei fals, există excepții.
    \pause
    \item X e mereu fals.
    \pause
    \item Nu se leagă afirmația (\textit{non sequitur}). (\textit{Dacă bei apă în fiecare zi, vei întâlni iubirea vieții tale.})
    \pause
    \item Legătura este una slabă, altele sunt mai importante. (\textit{Dacă dormi bine, vei avea musculatura dezvoltată.})
    \pause
    \item Afirmația este subiectivă, supusă interpretării.
    \pause
    \item Nu sunt suficiente informații.
  \end{itemize}
\end{frame}

\begin{frame}{Recap: Răspunsuri simple}
  \begin{itemize}
    \item X e mereu adevărat sau e mereu fals.
    \item Adevărul e undeva la mijloc.
    \item Totul e relativ.
  \end{itemize}
\end{frame}

\begin{frame}{De ce aleg oamenii răspunsuri simple?}
  \begin{itemize}
    \pause
    \item ,,eficiență'' / ,,lene'', psychological shortcuts
    \pause
    \item obișnuință, absență stimuli
    \pause
    \item echo chamber, efectul de bulă
    \pause
    \item presiune socială / public shaming / nevoia de apreciere
  \end{itemize}
\end{frame}

\begin{frame}{Ce e de făcut?}
  \begin{itemize}
    \pause
    \item Înțelege problema.
    \pause
    \item Lucrează la tine.
    \pause
    \item Lucrează la alții.
  \end{itemize}
\end{frame}

\begin{frame}{Problema: Fallacies, Biases}
  \centering
  \url{https://yourlogicalfallacyis.com/} \\
  \url{https://yourbias.is/}
\end{frame}

\begin{frame}{Nobody is safe}
  \textit{Smart people do stupid things because being smart doesn't make you wise, and wise people do foolish things because being wise doesn't mean emotions can't hijack your better judgement. \\
  In essence no one's perfect and everybody fucks up, just some are more prone to it than others.} \\
  \vspace{3mm}
  \hfill \textit{@TellYourSonThis}
\end{frame}

\begin{frame}{Soluția: Critical Thinking}
  \pause
  \textit{Dubito, ergo cogito; cogito, ergo sum.} \\
  \vspace{3mm}
  \hfill \textit{René Descartes} \\
  \vspace{1cm}
  \pause
  \textit{What society needs right now is an entire populace engaged in critical thinking. Question what you're being told, remain skeptical of the information you're hearing, and don't be afraid to analyze before believing. Treat the media like a cheating ex - keep your fucking guard up.} \\
  \vspace{3mm}
  \hfill \textit{The Captain (@sgrstk) on Twitter}
\end{frame}

\begin{frame}{Ușor de zis, greu de făcut}
  \begin{itemize}
    \pause
    \item multă răbdare
    \pause
    \item toleranță extremă a celorlalți; ascultă orice chiar dacă pare aberant, nu da unfollow, nu da block
    \pause
    \item citește opinii pe care le respingi
    \pause
    \item discuții în contradictoriu
    \pause
    \item emite opinii nepopulare, chiar contondente
    \pause
    \item spune ceva pentru că așa crezi, nu pentru că ,,așa e bine''
    \pause
    \item cât mai deschiși la a fi luați la mișto
    \pause
    \item ia-l pe ,,nu știu'' / ,,nu mă pricep'' în brațe
    \pause
    \item nu avea nici o problemă să spui ,,sunt prost / sunt proastă''
    \pause
    \item fii confortabil cu incertitudinile
  \end{itemize}
\end{frame}

\begin{frame}{Fii realist(ă)}
  \pause
  Vei dori să gândești critic, dar ești om. Nu vei putea filtra tot; nu ai timp / energie. Tot vei avea \ldots
  \begin{itemize}
    \pause
    \item opinii personale, ale tale, subiective
    \pause
    \item opinii pe care le consideri cvasi-universal adevărate
    \pause
    \item opinii pe baza cărora respingi pe alții, te consideri incompatibil
    \pause
    \item opinii pentru care militezi: \textbf{trebuie} să-i schimbi pe alții
  \end{itemize}
  \pause
  E OK. Câtă vreme ți le asumi, le poți afirma neagresiv, calm. Și nu sunt foarte multe.
\end{frame}

\begin{frame}{Lucrat la alții}
  \begin{itemize}
    \pause
    \item educă-te pe tine, admite opinii subiective și personale, nu le impune altora
    \pause
    \item apoi, educă-i pe alții
    \pause
    \item expunere constantă, cu lingurița
    \pause
    \item rabdare cu ei, nu te grăbi, durează mult (luni, ani)
    \pause
    \item nu fi arogant, nu vorbi de sus
    \pause
    \item corelație negativă cu vârstă și inteligență
    \pause
    \item luat la mișto, satiră, \textit{ridendo castigat mores}
    \pause
    \item accentul pe probleme, nu pe soluții
    \pause
    \item debate-urile sunt problematice, sunt prea punctuale și tind să fie agresive
  \end{itemize}
\end{frame}

\begin{frame}{Game of Thrones Pop Quiz}
  Cine se schimbă mai mult? Cine devine mai deschis, mai critic?
  \begin{itemize}
    \item Cersei Lannister
    \item Jaime Lannister
    \item Tyrion Lannister
  \end{itemize}
\end{frame}

\begin{frame}{}
  \centering
  \Large{Critical thinking != Criticism}
\end{frame}

\begin{frame}{Resurse}
  \begin{itemize}
    \item slide-urile prezentării: \url{https://www.slideshare.net/razvandeaconescu/}
    \item \url{https://yourlogicalfallacyis.com/}
    \item \url{https://yourbias.is/}
    \item The Captain (@sgrstk) (Twitter)
    \item Rolf Degen (@DegenRolf) (Twitter)
    \item Steve Stewart-Williams (@SteveStuWill) (Twitter)
    \item Jack Peach (@ThinkInPeach) (Twitter)
    \item Heterodox Academy (@HdxAcademy) (Twitter)
    \item Doru Căstăian (doru.castaian) (Facebook)
  \end{itemize}
\end{frame}

\end{document}
