\documentclass{beamer}

\usepackage[utf8x]{inputenc}
\usepackage[romanian]{babel}
\usepackage{color}
\usepackage{alltt}

%\usepackage{code/highlight}

\usepackage{hyperref}
\usepackage{verbatim}

\mode<presentation>
{ \usetheme{Berlin} }

\PrerenderUnicode{aâîțșĂÎÂȚȘ}

\title[Gestiunea laboratorului]{Soluții open-source pentru gestiunea sălilor
de laborator}
\institute{Informatica la Castel}
\author[Răzvan Deaconescu]{Răzvan Deaconescu \\
razvan.deaconescu@cs.pub.ro}
\date{29 august 2012}

\begin{document}

\frame{\titlepage}

\begin{frame}{Context}
  \begin{itemize}
    \item Facultatea de Automatică și Calculatoare, Departamentul de
      Calculataore
    \item 350 de studenți pe an, 3 serii de studenți pe an, 5 grupe pe serie,
      2 semigrupe pe grupă
    \item săli de laborator cu 10-18 sisteme de calcul (PC)
    \item în jur de 15 săli de laborator
    \item Linux folosit în sălile de laborator
    \item sesiuni de laborator a câte 2 ore
    \item în cazul unei alocări complete a unui an de studiu pe sală de
      laborator, este ocupată complet 8-20 în fiecare zi
  \end{itemize}
\end{frame}

\begin{frame}{Probleme și motivație (partea 1)}
  \begin{itemize}
    \item număr mare de sisteme care trebuie instalate în fiecare semestru
    \item se întâmplă să fie nevoie de actualizare pe toate sistemele
    \item în caz de examen de test practic la anumite discipline (USO, RL)
      este nevoie de instalarea aceleiași imagini în 4 săli (circa 80 de
      sisteme)
  \end{itemize}
\end{frame}

\begin{frame}{Soluție (partea 1)}
  \begin{itemize}
    \item clonarea discului
    \item folosirea rețelei locale
    \item un sistem este sursa de clonare iar celelalte sunt destinația
    \item \textit{along comes UDPcast}
  \end{itemize}
\end{frame}

\begin{frame}{UDPcast}
  \begin{itemize}
    \item imagine bootabilă personalizată de Linux
      \begin{itemize}
        \item \url{http://www.udpcast.linux.lu/}
      \end{itemize}
    \item un soi de \texttt{dd} în rețea
    \item permite configurarea unui \textit{sender/seeder} și a mai multor
      \textit{receiveri} în rețea
    \item folosește UDP și multicast pentru transferul informației
    \item permite transferului unui disc sau a unei partiții
  \end{itemize}
\end{frame}

\begin{frame}{Scenariu tipic de utilizare}
  \begin{itemize}
    \item se bootează de pe CD sau USB stick toate sistemele dorite din
      laborator (în general, se recomandă să fie identice)
    \item sunt selectate opțiunile de configurare a rețelei (DHCP sau static)
    \item se selectează partiția sau discul care să fie clonat(ă)
    \item se selectează forma de comprimare (recomandat lzop)
    \item se selectează \textbf{un} \textit{sender} și restul \textbf{receiveri}
    \item se realizează decuplarea rețelei de la restul rețelei mari
      (multicast înseamnă \ldots multicast)
    \item se declanșează transferul
    \item dacă totul merge bine se copiază datele cu viteza de circa 100Mbps
  \end{itemize}
\end{frame}

\begin{frame}{Bine de știut}
  \begin{itemize}
    \item se poate face fără probleme transferul de la un disc mai mic la un
      disc mai mare
    \item se poate face transferul de la un disc mai mare la unul mai mic doar
      dacă datele nu depășesc discul mai mic
    \item înainte de UDPcast, se recomandă folosirea \textit{zero fill} pe
      discul/partițiile de clonat pentru a mări viteza
      \begin{itemize}
        \item \texttt{dd if=/dev/zero of=zero-fill.tmp bs=8M}
        \item \texttt{sync}
        \item \texttt{rm zero-fill}
      \end{itemize}
    \item imaginea de kernel și cea de ramdisk pot fi copiate locale și apoi
      se poate boota folosind GRUB (fără a fi nevoie de un dispozitiv de boot
      -- CD-ROM sau USB stick)
      \begin{itemize}
        \item
          \url{https://koala.cs.pub.ro/gitweb/?p=systems/lab-infrastructure.git;a=blob;f=udpcast/install-udpcast;hb=HEAD}
      \end{itemize}
  \end{itemize}
\end{frame}

\begin{frame}{Probleme și motivație (partea a 2-a)}
  \begin{itemize}
    \item în cadrul fiecărei sesiuni de laborator se fac modificări
    \item unele laboratoare fac modificări la nivel de root (configurare
      servicii, configurare utilizatori)
    \item în cazul testelor practice, o nouă sesiune trebuie să găsească
      sistemul ,,curat''
    \item este recomandat (uneori obligatoriu) să nu fie vizibile modificările
      utilizatorilor
  \end{itemize}
\end{frame}

\begin{frame}{Soluție (partea a 2-a)}
  \begin{itemize}
    \item ,,freezing-ul'' discului sau partiției $\rightarrow$ toate
      modificările se vor dispune altundeva
    \item Există ceva existent? La momentul respectiv (2007), exista
      DeepFreeze.
      \begin{itemize}
        \item nu are suport bun pe Linux (acum deloc)
        \item este proprietar
      \end{itemize}
    \item Vlad Ureche \& Mircea Bardac to the rescue
      \begin{itemize}
        \item folosire unionfs și, mai târziu, aufs
        \item configurare în așa fel încât la repornire, sistemul de fișiere
          revine (în majoritate) la starea inițială
      \end{itemize}
  \end{itemize}
\end{frame}

\begin{frame}{Cum se configurează?}
  \begin{itemize}
    \item e nevoie de o partiție suplimentară pentru ,,backup-ul'' fiecărei
      partiții folosite
      \begin{itemize}
        \item ,,backup'' $\rightarrow$ zona în care se stochează modificările
      \end{itemize}
    \item scriptul \texttt{sysufs} este copiat în sistemul local de fișiere și
      apoi configurat să ruleze din \texttt{/etc/rc.local}
      \begin{itemize}
        \item \url{https://koala.cs.pub.ro/gitweb/?p=systems/lab-infrastructure.git;a=tree;f=freeze;hb=HEAD}
      \end{itemize}
  \end{itemize}
\end{frame}

\begin{frame}{aufs}
  \begin{itemize}
    \item folosit în distribuțiile Live
    \item union mount -- de obicei un sistem de fișiere montat read-only
      și celălalt montat read-write
  \end{itemize}
\end{frame}

\begin{frame}{Pași în folosirea scriptului \texttt{sysufs}}
  \begin{itemize}
    \item TODO
  \end{itemize}
\end{frame}

\begin{frame}{Efecte ,,secundare''}
  \begin{itemize}
    \item experiență legată de scripturile de inițializare
      (\textit{/etc/init.d})
    \item site-ul UDPcast nu mai este accesibil din rețeaua UPB
    \item totul este o partiție
    \item plăcile grafice \ldots NU
    \item porturile USB \ldots NU
  \end{itemize}
\end{frame}

\begin{frame}{Resurse utile}
  \begin{itemize}
    \item lab-infrastructure.git --
      \url{http://koala.cs.pub.ro/gitweb/?p=systems/lab-infrastructure.git;a=summary}
    \item UDP Cast -- \url{http://udpcast.linux.lu/}
    \item Clonezilla -- \url{http://www.clonezilla.org/}
    \item aufs -- \url{http://aufs.sourceforge.net/}
    \item DeepFreeze -- \url{http://www.faronics.com/enterprise/deep-freeze/}
  \end{itemize}
\end{frame}

\end{document}
