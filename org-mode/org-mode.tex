% vim: set tw=78 tabstop=4 shiftwidth=4 aw ai:
\documentclass{beamer}

\usepackage[utf8x]{inputenc}		% diacritice
\usepackage[romanian]{babel}
\usepackage{color}			% highlight
\usepackage{alltt}			% highlight

\usepackage{hyperref}			% folosiți \url{http://...}
					% sau \href{http://...}{Nume Link}
\usepackage{verbatim}

\usepackage{code/highlight} % highlight

\mode<presentation>
{ \usetheme{Berlin} }

% Încărcăm simbolurilor Unicode românești în titlu și primele pagini
\PreloadUnicodePage{200}

\title[Org-Mode]{Org-Mode: Your Life in Plain Text}
\subtitle{Lightning Talk}
\institute{Întâlnirile RLUG -- Aprilie 2011}
\author[Răzvan Deaconescu]{Răzvan Deaconescu\\
	razvan@rosedu.org}
\date{14 aprilie 2011}

\begin{document}

% Slide-urile cu mai multe părți sunt marcate cu textul (cont.)
\setbeamertemplate{frametitle continuation}[from second]

\frame{\titlepage}

\begin{frame}{Org-Mode}
  \begin{itemize}
    \item Emacs Mode
    \item notițe
    \item planificare
    \item TODO-uri
    \item menținut plain-text
  \end{itemize}
\end{frame}

\begin{frame}{Org-Mode Keywords}
  \begin{itemize}
    \item ierarhie
    \item tag-uri
    \item timestamp
    \item stare
  \end{itemize}
\end{frame}

\begin{frame}{Demo}
  \begin{itemize}
    \item Here we go!
  \end{itemize}
\end{frame}

\begin{frame}{Instalare și configurare}
  \begin{itemize}
    \item \texttt{apt-get install org-mode}
    \item în \texttt{$\sim$/.emacs}
  \end{itemize}
  \begin{beamerboxesrounded}[lower=block body,shadow=true]{}
    \scriptsize \input{code/dot-emacs-org-mode}
  \end{beamerboxesrounded}
\end{frame}

\begin{frame}{Extra}
  \begin{itemize}
    \item Thanks to Dusan Gabrijelcic!
      \begin{itemize}
        \item Să folosești Org-Mode nu e greu.
        \item Dificil este să îl folosești cât mai bine (potrivit nevoilor
        tale).
      \end{itemize}
    \item \url{https://github.com/hsitz/VimOrganizer}
  \end{itemize}
\end{frame}

\begin{frame}{Resurse utile}
	\begin{itemize}
        \item \url{http://orgmode.org/}
        \item \url{http://orgmode.org/worg/}
		\item \url{http://www.youtube.com/watch?v=oJTwQvgfgMM}
        \item \url{http://orgmode.org/orgcard.pdf}
    \end{itemize}
\end{frame}

\end{document}
