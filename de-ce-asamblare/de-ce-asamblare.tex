\documentclass{beamer}

% Romanian Language support
\usepackage{ucs}
\usepackage[utf8x]{inputenc}
\PrerenderUnicode{aâîțșĂÎÂȚȘ}
\usepackage[english,romanian]{babel}

\usepackage{hyperref}   % use \url{http://$URL} or \href{http://$URL}{Name}
\usepackage{verbatim}
\usepackage{underscore} % underscores need not be escaped
\usepackage{booktabs}   % nice looking tables
\usepackage{array}      % column size options in tables
\usepackage[normalem]{ulem}       % for striketrough text

\mode<presentation>
%{ \usetheme{Berlin} }

% Disable useless navigation symbols.
\setbeamertemplate{navigation symbols}{}
\setbeamertemplate{footline}[frame number]

\title[De ce asm?]{De ce să știi limbaj de asamblare?}
\institute{Informatica la Castel 2016 (Macea, Arad)}
\author[Răzvan Deaconescu]{Răzvan Deaconescu \\
razvan.deaconescu@cs.pub.ro}
\date{24 august 2016}

\begin{document}

\frame{\titlepage}

\begin{frame}{Exemple din securitatea ofensivă}
  \begin{itemize}
    \pause \item \textit{offensive security}
    \pause \item serviciu de tip \textit{hello} configurat pe mașină virtuală
    \pause \item \href{http://phrack.org/issues/49/14.html}{Smashing the Stack for Fun and Profit} (aleph1)
    \pause \item \url{https://www.exploit-db.com/local/}
    \pause \item \url{https://www.exploit-db.com/shellcode/}
    \pause \item \url{https://www.exploit-db.com/docs/39665.pdf}
  \end{itemize}
\end{frame}

\begin{frame}{Nevoia de optimizare}
  \centering
  \pause
  \textit{\href{http://c2.com/cgi/wiki?MakeItWorkMakeItRightMakeItFast}{Make it work. Make it right. Make it fast.}} \\
  \vspace{3mm}
  \hfill \textit{Kent Beck} \\
  \pause
  \vspace{1cm}
  \footnotesize{\url{http://osxr.org:8080/glibc/source/sysdeps/x86/bits/string.h}}
\end{frame}

\begin{frame}{Nevoia de debugging}
  \centering
  \pause printf debugging\\
  \pause valori variabile\\
  \pause funcții apelate (fluxul programului)\\
  \pause \textbf{valori din memorie}\\
  \pause \textbf{valorile registrelor procesorului}\\
  \pause \textbf{codul rulat de procesor}\\
\end{frame}

\begin{frame}{Dezvoltare low-level}
  \begin{itemize}
    \pause \item sisteme încoporate (\textit{embedded systems})
      \begin{itemize}
        \pause \item \textit{Board Support Package}: bootloader, kernel, drivere
      \end{itemize}
    \pause \item software de sistem (\textit{system software})
    \pause \item dezvoltare de kernel (\textit{kernel development})
  \end{itemize}
\end{frame}

\begin{frame}{Înainte de a începe}
  \centering
  \pause mai mult citim decât scriem \\
  \pause ca să scrii povești bune trebuie să citești povești bune \\
  \pause ca să scrii cod bun trebuie să citești cod bun \\
  \pause mai mult ca orice, limbajul de asamblare mai mult se citește \\
  \vspace{1cm}
  \pause \textbf{Învățăm limbaj de asamblare ca să înțelegem, mult mai rar să scriem.} \\
  \hfill \textit{R. Deaconescu, 2016, Macea, Arad}
\end{frame}

\begin{frame}{De ce e mai ușor?}
  \begin{itemize}
    \item mașini virtuale
    \item emulatoare
    \item documentație
  \end{itemize}
\end{frame}

\begin{frame}{Funcționarea sistemului de calcul}
  \begin{figure}
    \centering
    \includegraphics[width=0.7\textwidth]{img/functionarea-sistemului-de-calcul.pdf}
  \end{figure}
\end{frame}

\begin{frame}{Bazele limbajului de asamblare}
  \begin{itemize}
    \pause \item datele rezidă în memorie
    \pause \item datele sunt aduse din memorie în registrele procesorului
    \pause \item procesorul face operații cu datele din registre
    \pause \item rezultatele operațiilor sunt scrise înapoi în memorie
    \pause \item unele informații pot fi preluate și transmise către dispozitive de intrare/ieșire
  \end{itemize}
\end{frame}

\begin{frame}{Operații de asamblare}
  \begin{figure}
    \centering
    \includegraphics[width=0.9\textwidth]{img/operatii-de-asamblare.pdf}
  \end{figure}
\end{frame}

\begin{frame}{Exemple în limbaj de asamblare}
  \begin{itemize}
    \item \texttt{asm/hello.asm}
    \item \texttt{asm/print\_n.asm}
    \item \texttt{asm/print\_n\_offset.asm}
  \end{itemize}
\end{frame}

\begin{frame}{Compiler Explorer}
  \begin{itemize}
    \item \url{https://gcc.godbolt.org/}
  \end{itemize}
\end{frame}

\begin{frame}{Înțelegerea sistemului de calcul}
  \centering
  \pause De ce învață oamenii obișnuiți să cânte la instrumente muzicale? \\
  \pause De ce fac oamenii obișnuiți sport în mod activ? \\
  \pause De ce le place oamenilor (în special copiilor) să demonteze lucruri? \\
  \vspace{5mm}
  \pause \textbf{Mastery}, Autonomy and Purpose (Dan Pink: Drive)\\
  \vspace{5mm}
  \pause curiozitate și expertiză
\end{frame}

\begin{frame}{John ``maddog'' Hall: The Third Language}
  \begin{itemize}
    \item \footnotesize{\url{https://www.lpi.org/the-third-language/}}
    \item \footnotesize{\url{http://www.techworld.com/operating-systems/john-maddog-hall-why-raspberry-pi-is-only-beginning-3453073/2/}}
  \end{itemize}
\end{frame}

\begin{frame}{Joel Spolsky: Law of Leaky Abstractions}
  \begin{itemize}
    \item \footnotesize{\url{http://www.joelonsoftware.com/articles/LeakyAbstractions.html}}
  \end{itemize}
\end{frame}

\begin{frame}{Parcurgerea unei matrice}
  \centering
  Demo: \textit{walk\_matrix\_row\_major.c}, \textit{walk\_matrix\_column\_major.c}
\end{frame}

\begin{frame}{Shlemiel the Painter's Algorithm}
  \begin{itemize}
    \item Joel Spolsky: Back to Basics
      \begin{itemize}
        \item \footnotesize{\url{http://www.joelonsoftware.com/articles/fog0000000319.html}}
      \end{itemize}
  \end{itemize}
\end{frame}

\begin{frame}{Povești, povestiri, amintiri}
  \begin{itemize}
    \item Minunata călătorie a lui Răzvan ,,Senpai'' Deaconescu în lumea calculatoarelor
  \end{itemize}
\end{frame}

\begin{frame}{Cum învăț/cum aprofundez}
  \begin{itemize}
    \pause \item Nu învăța ceva ca să înveți ceva / că e bine / că se caută / că vrei să știi.
    \pause \item Pune-ți obiective și învață ca o consecință. \textit{Means to an end}.
    \pause \item competiții de tip CTF (Capture the Flag)
    \pause \item site-uri de tip wargames
    \pause \item profiling la aplicații
    \pause \item programează pe platforme ARM (sau MIPS): Raspberry Pi
    \pause \item Randal E. Bryant, David R. O'Hallaron: Computer Systems: A Programmer's Perspective
  \end{itemize}
\end{frame}

\begin{frame}{La final}
  \begin{itemize}
    \pause \item Învață în profunzime.
    \pause \item Strică lucrurile.
    \pause \item Detaliile fac diferența.
    \pause \item Răbdarea este o virtute.
    \pause \item Laborator de introducere în limbaj de asamblare: joi, 25 august 2016, ora 21:00
  \end{itemize}
\end{frame}

\begin{frame}{Mulțumesc!}
  \begin{itemize}
    \item \tiny{\url{http://www.slideshare.net/razvandeaconescu/de-ce-s-tii-limbaj-de-asamblare}}
    \item \tiny{\url{https://github.com/razvand/slides/tree/master/de-ce-asamblare}}
  \end{itemize}
\end{frame}

\end{document}
