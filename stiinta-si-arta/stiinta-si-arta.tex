\documentclass{beamer}

% Romanian Language support
\usepackage{ucs}
\usepackage{alltt}
\usepackage[utf8x]{inputenc}
\PrerenderUnicode{aâîțșĂÎÂȚȘ}
\usepackage[english,romanian]{babel}

\usepackage{hyperref}   % use \url{http://$URL} or \href{http://$URL}{Name}
\usepackage{verbatim}
\usepackage{underscore} % underscores need not be escaped
\usepackage{booktabs}   % nice looking tables
\usepackage{array}      % column size options in tables
\usepackage[normalem]{ulem}       % for striketrough text

\mode<presentation>
%{ \usetheme{Berlin} }

% Disable useless navigation symbols.
\setbeamertemplate{navigation symbols}{}

\title[Știință și artă]{Știință și artă}
\institute{CDL, ROSEdu}
\author[Răzvan Deaconescu]{Răzvan Deaconescu \\
razvan@rosedu.org}
\date{13 mai 2017}

\begin{document}

\frame{\titlepage}

\begin{frame}{De ce există prezentări și discursuri?}
  \begin{itemize}
    \item conținutul este una
    \item livrarea și experiența personală sunt alta
  \end{itemize}
\end{frame}

\begin{frame}{Ce anume face din cineva un programator bun?}
\end{frame}

\begin{frame}{Ce anume face din cineva un programator excepțional?}
\end{frame}

\begin{frame}{Ce e mai bine?}
  \begin{center}
    \pause
    rigoare sau creativitate \\
    \vspace{0.5cm}
    \pause
    cunoștințe sau intuiție \\
    \vspace{0.5cm}
    \pause
    reguli sau libertate\\
    \vspace{1cm}
    \pause
    \Large{depinde}
  \end{center}
\end{frame}

\begin{frame}{Știință și artă}
  \begin{columns}
    \begin{column}{0.5\textwidth}
      \begin{itemize}
        \item rigoare
        \item reguli
        \item determinism
        \item rezultate
        \item cunoștințe
        \item istorie
        \item obiectivitate
      \end{itemize}
    \end{column}
    \begin{column}{0.5\textwidth}
      \begin{itemize}
        \item creativitate
        \item libertate
        \item noi posibilități
        \item călătorie
        \item joc
        \item noutate
        \item subiectivitate
      \end{itemize}
    \end{column}
  \end{columns}
\end{frame}

\begin{frame}{Poetry. Man and machine}
  \textit{We don't read and write poetry because it's cute. We read and write poetry because we are members of the human race. And the human race is filled with passion. And medicine, law, business, engineering, these are noble pursuits and necessary to sustain life. But poetry, beauty, romance, love, these are what we stay alive for.}\\
  \vspace{0.5cm}
  \hfill{John Keating, \textit{Dead Poets Society}}
\end{frame}

\begin{frame}{Programarea ca știință}
  \begin{itemize}
    \item TODO
  \end{itemize}
\end{frame}

\begin{frame}{Programarea ca artă}
  \begin{itemize}
    \item TODO
  \end{itemize}
\end{frame}

\begin{frame}{Pick-Up Artists (PUA)}
  \begin{itemize}
    \item TODO
  \end{itemize}
\end{frame}

\begin{frame}{The Game}
  \textit{What you must learn is that these rules are no different that the rules of a computer system. Some of them can be bent. Others can be broken.}\\
  \vspace{0.5cm}
  \hfill{Morpheus, \textit{The Matrix}}
\end{frame}

\begin{frame}{Știință sau artă}
  \begin{itemize}
    \item falsă dihotomie, \textit{chicken and egg}
    \item la pachet
  \end{itemize}
\end{frame}

\begin{frame}{De ce (nu) știință?}
  \begin{itemize}
    \item bun, dar niciodată excelent
  \end{itemize}
\end{frame}

\begin{frame}{De ce (nu) artă?}
  \begin{itemize}
    \item rezultate senzaționale dar impredictibile
  \end{itemize}
\end{frame}

\begin{frame}{Alte arte}
  \begin{itemize}
    \pause
    \item conversației
    \item persuasiunii, influenței, manipulării
    \item programării
    \item predatului
    \item politicii
    \item managementului
    \item conducerii
    \item războiului
    \item afacerilor
    \item povestirii
  \end{itemize}
\end{frame}

\begin{frame}{Omul cu 1000 de fețe}
  \begin{itemize}
    \item autoritar sau consensual
    \item serios sau amuzant/relaxat
    \item detașat sau implicat
    \item idealist sau realist
    \item răbdător sau expeditiv
    \item intransigent sau înțelegător
    \item individualist sau altruist
  \end{itemize}
\end{frame}

\begin{frame}{,,Nu mă pricep la asta''}
  \begin{itemize}
    \item TODO
  \end{itemize}
\end{frame}

\begin{frame}{Artistul din tine}
  \begin{itemize}
    \item naturalețe
    \item autenticitate
    \item consecvență
  \end{itemize}
\end{frame}

\begin{frame}{Calități pentru un artist}
  \begin{itemize}
    \item observabilitate
    \item adaptabilitate
    \item creativitate
  \end{itemize}
\end{frame}

\begin{frame}{Îmbinare știință cu artă}
  \begin{itemize}
    \item e o artă :-))
    \item 75\% știință, 25\% artă
  \end{itemize}
\end{frame}

\begin{frame}{TODO}
  \begin{itemize}
    \item citește
    \item meditează
    \item greșește
  \end{itemize}
\end{frame}

\begin{frame}{De final}
  \begin{center}
    Învață regulile. Urmărește regulile. Adaptează regulile. Redefinește regulile.
  \end{center}
\end{frame}

\end{document}
