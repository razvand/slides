\documentclass{simple}

\title[Știință și artă]{Știință și artă}
\institute{CDL, ROSEdu}
\author[Răzvan Deaconescu]{Răzvan Deaconescu \\
razvan@rosedu.org}
\date{13 mai 2017}

\begin{document}

\frame{\titlepage}

\begin{frame}{De ce există prezentări și discursuri?}
  \begin{itemize}
    \pause
    \item Scott Meyers: \textit{Things that matter}
    \pause
    \item conținutul este una
    \pause
    \item livrarea și experiența personală sunt alta
  \end{itemize}
\end{frame}

\begin{frame}{Ce anume face din cineva un programator bun?}
  \begin{itemize}
    \pause
    \item cunoștințe
    \item experiență anterioară
    \item rigoare
    \item controlul utilitarelor
  \end{itemize}
\end{frame}

\begin{frame}{Ce anume face din cineva un programator excepțional?}
  \begin{itemize}
    \pause
    \item vede pădurea, nu doar copacii
    \item creativitate
    \item disciplină
    \item intuiție
    \item învățare continuă
    \item vede și utilizatorii, nu doar tehnologia
  \end{itemize}
\end{frame}

\begin{frame}{Ce e mai bine?}
  \begin{center}
    \pause
    rigoare sau creativitate \\
    \vspace{0.5cm}
    \pause
    cunoștințe sau intuiție \\
    \vspace{0.5cm}
    \pause
    reguli sau libertate\\
    \vspace{1cm}
    \pause
    \Large{depinde}
  \end{center}
\end{frame}

\begin{frame}{Știință și artă}
  \begin{columns}
    \begin{column}{0.5\textwidth}
      \begin{itemize}
        \item rigoare
        \item reguli
        \item determinism
        \item rezultate
        \item cunoștințe
        \item istorie
        \item obiectivitate
      \end{itemize}
    \end{column}
    \begin{column}{0.5\textwidth}
      \begin{itemize}
        \item creativitate
        \item libertate
        \item noi posibilități
        \item călătorie
        \item joc
        \item noutate
        \item subiectivitate
      \end{itemize}
    \end{column}
  \end{columns}
\end{frame}

\begin{frame}{Poetry. Man and machine}
  \textit{We don't read and write poetry because it's cute. We read and write poetry because we are members of the human race. And the human race is filled with passion. And medicine, law, business, engineering, these are noble pursuits and necessary to sustain life. But poetry, beauty, romance, love, these are what we stay alive for.}\\
  \vspace{0.5cm}
  \hfill{John Keating, \textit{Dead Poets Society}}
\end{frame}

\begin{frame}{Programarea ca știință}
  \begin{itemize}
    \pause
    \item limbaje de programare: sintaxă, semantică
    \item bune practici
    \item tehnici de programare
    \item design patterns
    \item coding style
  \end{itemize}
\end{frame}

\begin{frame}{Programarea ca artă}
  \begin{itemize}
    \pause
    \item the best tool for the best job
    \item trade off-uri: limbaj de programare, framework-uri, arhitectură
    \item accentul pe utilizator
    \item hacking
    \item frumusețea programului
  \end{itemize}
\end{frame}

\begin{frame}{Pick-Up Artists (PUA)}
  \begin{itemize}
    \pause
    \item ,,artiștii agățării''
    \item reguli/recomandări: \textit{inner game}, \textit{outer game}, \textit{natural game}
    \item depinde de persoană
    \item \textit{confidence}
  \end{itemize}
\end{frame}

\begin{frame}{The Game}
  \textit{What you must learn is that these rules are no different that the rules of a computer system. Some of them can be bent. Others can be broken.}\\
  \vspace{0.5cm}
  \hfill{Morpheus, \textit{The Matrix}}
\end{frame}

\begin{frame}{Știință sau artă}
  \begin{itemize}
    \pause
    \item falsă dihotomie, \textit{chicken and egg}
    \item la pachet
  \end{itemize}
\end{frame}

\begin{frame}{De ce (nu) știință?}
  \begin{itemize}
    \pause
    \item bun, dar niciodată excelent
    \item reguli clare
    \item predictibilitate
    \item se poate învăța din surse existente
    \item durează
  \end{itemize}
\end{frame}

\begin{frame}{De ce (nu) artă?}
  \begin{itemize}
    \pause
    \item rezultate incerte (posibil extraordinare, posibil nu)
    \item se învață doar practicând
    \item subiectivă, ține de fiecare persoană
    \item durează
  \end{itemize}
\end{frame}

\begin{frame}{Alte arte}
  \begin{itemize}
    \pause
    \item conversației
    \item persuasiunii, influenței, manipulării
    \item programării
    \item predatului
    \item politicii
    \item managementului
    \item conducerii
    \item războiului
    \item afacerilor
    \item povestirii
  \end{itemize}
\end{frame}

\begin{frame}{Omul cu 1000 de fețe}
  \begin{itemize}
    \pause
    \item autoritar sau consensual
    \pause
    \item serios sau amuzant/relaxat
    \pause
    \item detașat sau implicat
    \pause
    \item idealist sau realist
    \pause
    \item răbdător sau expeditiv
    \pause
    \item intransigent sau înțelegător
    \pause
    \item individualist sau altruist
  \end{itemize}
\end{frame}

\begin{frame}{,,Nu mă pricep la asta''}
  \begin{itemize}
    \pause
    \item ,,utter and complete BS'' (Răzvan Deaconescu, 13 mai 2017)
    \pause
    \item nu vei fi excelent peste tot, dar poți ajunge mai bun
    \pause
    \item ,,be yourself'' $\rightarrow$ ,,be your best self''
    \pause
    \item do, fail, learn
  \end{itemize}
\end{frame}

\begin{frame}{Artistul din tine}
  \begin{itemize}
    \pause
    \item naturalețe
    \pause
    \item autenticitate
    \pause
    \item consecvență
  \end{itemize}
\end{frame}

\begin{frame}{Calități pentru un artist}
  \begin{itemize}
    \pause
    \item observabilitate
    \pause
    \item adaptabilitate
    \pause
    \item creativitate
  \end{itemize}
\end{frame}

\begin{frame}{Îmbinare știință cu artă}
  \begin{itemize}
    \item e o artă :-))
    \item 75\% știință, 25\% artă
  \end{itemize}
\end{frame}

\begin{frame}{TODO}
  \begin{itemize}
    \pause
    \item citește
    \pause
    \item meditează
    \pause
    \item greșește
  \end{itemize}
\end{frame}

\begin{frame}{De final}
  \begin{center}
    \pause
    Învață regulile.\\
    \vspace{0.5cm}
    \pause
    Urmărește regulile.\\
    \vspace{0.5cm}
    \pause
    Adaptează regulile.\\
    \vspace{0.5cm}
    \pause
    Redefinește regulile.
  \end{center}
\end{frame}

\end{document}
