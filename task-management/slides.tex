\documentclass{simple}

\title[Task Management]{Task Management for the Daily Workaholic}
\institute{Întâlnirile lunare RLUG -- August 2011}
\author[Răzvan Deaconescu]{Răzvan Deaconescu \\
    razvan@rosedu.org}
\date{11 august 2011}

\begin{document}

\frame{\titlepage}

\begin{frame}{Pentru task-uri proprii: Keep It Simple}
  \begin{itemize}
    \item \textbf{(OK) task: ceva de făcut (TODO)}
    \item (Not OK) task: ceva de făcut, deadline, prioritate, resurse folosite,
    proiect în care se încadrează, persoane implicate, legătură cu alte task-uri
    \item Atenție la PIMs, issue trackers, GTD-compliant apps
      \begin{itemize}
        \item consumi timp ca să te organizezi
      \end{itemize}
  \end{itemize}
\end{frame}

\begin{frame}{Task-uri ``imediate'' vs. task-uri ``amânabile''}
  \begin{itemize}
    \item \textbf{task-urile imediate trebuie avute în permanență în prim plan}
      \begin{itemize}
        \item foaie de hârtie
        \item sticky notes
        \item fișier text
        \item știi ce ai de făcut, nu e nevoie de detalii (fără priorități,
        deadline, proiecte legate, rezultate)
      \end{itemize}
    \item task-urile amânabile pot fi trecute într-un sistem mai ``complex''
      \begin{itemize}
        \item org-mode
        \item Taskwarrior
        \item GTD-compliant
        \item nu trebuie consultate în permanență, pot să nu fie imediat
        accesibile
      \end{itemize}
  \end{itemize}
\end{frame}

\begin{frame}{Deadline-uri}
  \begin{itemize}
    \item Douglas Adams Quotes -- \textit{I love deadlines. I like the
    whooshing sound they make as they fly by.}
    \item deadline-urile sunt importante, dar nu atât de importante; mai
    degrabă orientative
    \item o planificare bună înseamnă mai mult decât deadline-uri
    \item deadline-uri realiste, planificare riguroasă, versiuni draft,
    remindere
  \end{itemize}
\end{frame}

\begin{frame}{Ca să nu uiți}
  \begin{itemize}
    \item ``trebuie să fac asta, dar sunt prins'' sau ``ar fi bine să fac
    asta'' sau ``n-ar strica așa ceva'' -- \textbf{note it down}
    \item fă o listă cu ceea ce ai vrea să faci și ce probleme ai întâlnit și
    nu poți aloca timp să le faci pe moment
    \item când prinzi timp (tren, metrou, nu ai curent, nu ai Internet)
    gândește-te la ce ai de făcut, ce ar fi bine să faci și scrie-le undeva
    \item începe ziua cu lista de task-uri, încheie ziua cu ce ai făcut
      \begin{itemize}
        \item Today was a good day for \ldots (insert random word here)!
      \end{itemize}
  \end{itemize}
\end{frame}

\begin{frame}{Când lucrezi cu alții}
  \begin{itemize}
    \item mai mult de muncă
    \item \textbf{descriere clară a task-ului (ce? cum?)}
    \item recomandare resurse, link-uri etc.
    \item descriere clară a interacțiunii cu alt task-uri și activități (de
    unde iau input? unde dau output?)
    \item deadline -- orientativ, ca să pui presiune; eventual versiune draft
    \item buzz înainte de deadline (reminder)
    \item notificare automată la creare, update, apropierea deadline-ului
    \item folosire aplicație dedicată (assigned task, issue tracker etc.)
    \item monitorizare: \% done, status
  \end{itemize}
\end{frame}

\begin{frame}{Resurse utile (?)}
  \begin{itemize}
    \item David Allen -- Getting Things Done
    \item \url{http://lifehacker.com/}
    \item \url{http://www.thesimpledollar.com/}
  \end{itemize}
\end{frame}

\end{document}
